\documentclass[11pt,a4paper]{article}
\usepackage[utf8]{inputenc}
\usepackage[T1]{fontenc}
\usepackage[english, french]{babel}
\usepackage{fancyhdr}
\usepackage[margin=.7in]{geometry}
\usepackage{exsheets}
\usepackage{enumerate}

\usepackage{./Style/linearProgramsStyle} % This is a set of commands for maths content.

\pagestyle{fancy}

\renewcommand{\headrulewidth}{2pt}
\fancyhead[L]{EPITA\_ING1\_2020\_S5\_PROL}
\fancyhead[R]{Mars 2018}

\fancyfoot[C]{\textbf{\thepage}}
\fancyfoot[L]{}

\pretitle{\vspace{-.5\baselineskip} \begin{center}}
\title{%
  { \huge Programmation linéaire}\\ %
  \vspace{.5\baselineskip}
  {\Large \textit{Épreuve complémentaire} }
}
\posttitle{
\end{center}
  \begin{flushleft}
    \vspace{2\baselineskip} \textit{
      \!\!\emph{Durée de l'épreuve: 1h30.}\\
      \! \emph{Les documents ne sont pas autorisés}  \\
      Les caclulatrices non programmables sont autorisées.
    }
  \end{flushleft}
  \rule{\textwidth}{1.5pt}
  \vspace{-5\baselineskip}
}
\author{}
\date{}

\pdfinfo{
   /Author (Bashar Dudin)
   /Title  (Exam PROL-- 2018)
   /Subject (Programmation lineaire)
}

\begin{document}

\maketitle\thispagestyle{fancy}

On considère les programmes linéaires suivants respectivement nommés
$(A)$, $(B)$, $(C)$, $(D)$, $(E)$ et $(F)$.
\[
(A) \;
\left\{
  \begin{linearProg}{
      maximize
    }{
      $3x_1 + 2x_2 + 3x_3$
    }{
      \systeme{
        -x_1 - 2x_2 - 3x_3 \leq 5,
        2x_1 + x_2 + x_3 \leq 3,
        -x_1 - x_2 + x_3 \leq 1
      }
    }{
      $x_1$, $x_2$, $x_3 \geq 0.$
    }
  \end{linearProg}
\right.
\qquad (B)\;
\left\{
  \begin{linearProg}{
      minimize
    }{
      $ x_1 + x_2 - x_3$
    }{
      \systeme{
        -x_1 - 2x_2 + 3x_3 \leq 1,
        - x_1 - x_2 + 2x_3 \geq 2
      }
    }{
      $x_1$, $x_2$, $x_3 \geq 0.$
    }
  \end{linearProg}
\right.
\]
\[
(C) \;
\left\{
  \begin{linearProg}{
      maximize
    }{
      $3x_1 + 2x_2 + 3x_3$
    }{
      \systeme{
        -x_1 - 2x_2 - 3x_3 \leq 5,
        2x_1 + x_2 + x_3 \leq 3,
        -x_1 - x_2 + x_3 \leq 1
      }
    }{
      $x_1$, $x_2 \geq 0.$
    }
  \end{linearProg}
\right.
\qquad (D) \;
\left\{
  \begin{linearProg}{
      maximize
    }{
      $-x_1 - x_2 + x_3$
    }{
      \systeme{
        -x_1 + 2x_2 - 3x_3 + x_4 = 1,
         x_1 + x_2 - 2x_3 \leq -2
      }
    }{
      $x_1$, $x_2$, $x_3$, $x_4 \geq 0.$
    }
  \end{linearProg}
\right.
\]
\[
\qquad (E)\;
\left\{
  \begin{linearProg}{
      maximize
    }{
      $ -x_1 - x_2 + x_3$
    }{
      \systeme{
        -x_1 - 2x_2 + 3x_3 \leq 1,
        - x_1 - x_2 + 2x_3 \geq 2
      }
    }{
      $x_1$, $x_2$, $x_3 \geq 0.$
    }
  \end{linearProg}
\right.
\qquad (F)\;
\left\{
  \begin{linearProg}{
      maximize
    }{
      $6x_1 + 4x_2 + 6x_3$
    }{
      \systeme{
        -x_1 - 2x_2 - 3x_3 \leq 5,
        2x_1 + x_2 + x_3 \leq 3,
        x_1 + x_2 - x_3 \geq -1
      }
    }{
      $x_1$, $x_2 \geq 0.$
    }
  \end{linearProg}
\right.
\]

\section{Équivalence et dualité}

\begin{question}{3}
  Grouper les programmes linéaires précédents suivant leurs classes
  d'équivalences ; deux programmes sont dans la même classe s'ils sont
  équivalents.
\end{question}

\begin{question}{2}
  Donner les tableaux associés à $(F)$ et $(B)$.
\end{question}

\begin{question}{2}
  Quels sont les programmes duaux de $(A)$ et de $(E)$?
\end{question}

\begin{question}{2}
  Deviner des solutions admissibles de $(A)$ et de $(D)$.
\end{question}

\section{Algorithme du simplexe}

\begin{question}{4}
  Résoudre le programme linéaire $(A)$ en suivant l'algorithme du
  simplex.
\end{question}

\begin{question}{2}
  À partir d'une solution de $(A)$ donner un point optimal de son
  dual. Quelle est sa valeur objective?
\end{question}

\begin{question}{5}
  Résoudre le programme duale de $(E)$.
\end{question}

\end{document}

%%% Local Variables:
%%% mode: latex
%%% TeX-master: t
%%% End:
