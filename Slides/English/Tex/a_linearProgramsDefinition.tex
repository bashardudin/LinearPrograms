\documentclass[aspectratio = 169]{beamer}

\usepackage[utf8]{inputenc} % Character encoding.

\pdfinfo{
   /Author (Bashar Dudin)
   /Title  (What is a linear program?)
   /Subject (Linear Programs)
}

\usepackage{./Style/linearProgramsBeamer} % This is extra styling for Beamer environments.
\usepackage{./Style/linearProgramsStyle} % This is a set of commands for maths content.

%-------------------------------------------------------------------------------
%   TITLE PAGE
%-------------------------------------------------------------------------------

\author[BD]{Bashar Dudin}

\institute[]{EPITA}

\title{Linear Programs} %
\subtitle{What is a linear program?}

%-------------------------------------------------------------------------------
%   DOCUMENT BODY
%-------------------------------------------------------------------------------

\begin{document}

\begin{frame}[plain]
\titlepage % Print the title page as the first slide
\end{frame}


\begin{frame}{Dummy Example I | Plane Supply}
  An airline is opening a new route from city $A$ to city $C$
  transitting through city $B$. It can fill it's fuel tank at both
  $A$ and $B$ airports. Price of fuel at $A$ is $3$ dollars per
  galon and $2$ gallons at $B$. Aim of airline is to minimize the
  total cost of fuel over the trip from $A$ to $C$ while making sure
  that the conditions hereby are satisfied:
  \begin{itemize}
  \item<2-> Safety regulations expect you to have at least $1k$
    gallons to take off from $A$ to $B$.
  \item<3-> Safety regulations expect you to have at least $3k$
    gallons on your trip from $B$ to $C$. Note that you do have tank
    half full at $B$ ; having more fuel makes plane heavier which
    consumes more fuel.
  \item<4-> Your maximum tank capacity is $6k$ gallons.
  \item<5-> To keep fuel prices at their current price in both $A$
    and $B$ airline has to buy a total higher than $5k$ gallons over
    the whole trip.
  \end{itemize}
\end{frame}

\begin{frame}{Dummy Example I | Plane Supply}
  There are a couple of steps to go through in order to tackle the
  previous problem. One first starts by modeling the problem. Let
  $x_1$ and $x_2$ be the amount of fuel in kilo-gallons respectively
  taken at $A$ and $B$.
  \begin{itemize}
  \item Represent problem's constraints as a geometric region in
    $2$-dimensional space.
  \item What is the function you're trying to minimize?
  \item Given an \textit{ad hoc} cost, single out the possible values
    for $x_1$ and $x_2$ having that such cost.
  \item Does it help to figure out a strategy to figure out the
    minimal cost?
  \end{itemize}
  \pause We've grazed here the geometric approach to solving linear
  programs in low dimensions. It is not central in this course, but it
  is the starting point for solving much harder optimisation problems.
\end{frame}


\begin{frame}{Dummy Example II | An Election Issue}
  \begin{onlyenv}<1>
    This is taken from \textbf{Introduction to Algorithms} by
    \emph{Cormen, Lieserson, Rivest and Stein} :
    \begin{quotation}
      Suppose that you are a politician trying to win an
      election. Your district has three different types of areas:
      urban, suburban, and rural. These areas have respectively:
      100.000, 200.000 and 50.000 registered voters. Although not
      all the registered voters actually go to the polls, you
      decide that to govern effectively, you would like at least
      half the registered voters in each of the three regions to
      vote for you. You are honorable and would never consider
      supporting policies in which you do not believe. You
      realize, however, that certain issues may be more effective
      in winning votes in certain places. Your primary issues are
      building roads, gun control, farm subsidies, and a gasoline
      tax dedicated to improve public transit. According to your
      campaign staff's research, you can estimate how many voters
      you win or lose from each population segment by spending
      1.000€ on advertising on each issue.
    \end{quotation}
  \end{onlyenv}
  \begin{onlyenv}<2>
    \begin{figure}
      \begin{tabular}{l|ccc}
        policy & urban & suburban & rural \\
        \hline
        build road & -2 & 5 & 3 \\
        gun control & 8 & 2 & -5 \\
        farm subsidies & 0 & 0 & 10 \\
        gasoline tax & 10 & 0 & -2
      \end{tabular}
    \end{figure}
    \begin{quotation}
      In this table, each entry indicates the number of
      \emph{thousands} of either urban, suburban, or rural voters
      who would be won over by spending 1.000€ on advertising on
      support of a particular issue. Negative entries denote votes
      that would be lost. Your task is to figure out the minimum
      amount of money that you need to spend in order to win
      50.000 urban votes, 100.000 suburban votes, and 25.000 rural
      votes.
    \end{quotation}
  \end{onlyenv}
\end{frame}

\begin{frame}{Modeling}
    \begin{onlyenv}<1-2>
      The total cost of the campaign is equal to the cost of
      advertisement on each given issue. Let us call $x_1$, $x_2$,
      $x_4$ and $x_5$ the respective costs of advertisement on building
      roads, gun control, farm subsidies and gasoline tax. Each such
      cost has to be non-negative. Thus
        \begin{equation}
            \label{eq:positivityCondition}
            x_1, x_2, x_3, x_4 \geqslant 0.
        \end{equation}
        The first column gives the following constraint on each of the
        costs
        \begin{equation}
            \alt<1>{-2x_1 + 8x_2 + 0x_3 + 10x_4 \geqslant 50}{\textcolor{orange}{-2x_1 + 8x_2 + 10x_4 \geqslant 50}}
        \end{equation}
        The LHS corresponds to the fact your aim is to have more than
        50.000 voters from urban areas.
    \end{onlyenv}
    \begin{onlyenv}<3>
      Each one of the table's columns gives such a linear constraint
      on the set of costs. One can sum the issue at hand in the
      following way:
        \begin{figure}
            \begin{linearProg}{
                minimize
                }{
                $x_1+x_2+x_3+x_4$
                }{
                \systeme{-2x_1 + 8x_2 + 0x_3 + 10x_4 \geq  50,
                          5x_1 + 2x_2 + 0x_3 + 0x_4\geq 100,
                          3x_1 - 5x_2 + 10x_3 - 2x_4 \geq 25}
                }{
                $x_1$, $x_2$, $x_3$, $x_4 \geq 0$
                }
            \end{linearProg}
        \end{figure}
    This is now our first example of a linear program.
    \end{onlyenv}
\end{frame}

\begin{frame}{General Definition}
\begin{defn}
  A linear program is a problem of either minimizing or maximizing a
  linear function subject to a number of linear constraints.
\end{defn}
\alt<3>{An \alert{\emph{affine function}}}{A \emph{linear
    function}} in a $k$ variables (for $k \in \N^*$) is a function $f$
having the expression
    \[
\        f(x_1, x_2, \ldots, x_k) = \uncover<3>{\alert{c_0 + }} c_1x_1 + c_2x_2 + \cdots + c_kx_k
    \]
    for $k$ real numbers $c_1$, $\ldots$, $c_k$. A \emph{linear
      constraint} is either an equality or a (large) inequality, only
    containing affine expressions.

    \pause
    The \alt<3>{\alert{\textit{affine function}}}{\textit{linear
        function}} we're trying to minimize or maximize in a linear
    program is called the \emph{objective
      function}$^{\spadesuit}$. When the objective function is to be
    minimized the linear program is said to be a minimization
    program. It is a maximization program otherwise.
\end{frame}

\begin{frame}{General Definition}
  Given a linear program, our goal is to find solutions to the linear
  constraints such that the objective function has optimal value: a
  maximal value if we have a maximization linear program or a minimal
  one for minimization programs. A value taken by an objective
  function is called an \emph{objective value}.

  \pause
  A solution to the linear constraints, even if it doesn't minimize
  nor maximize the objective function, is called a \emph{feasible
    solution}. Most algorithms solving linear programs iteratively
  build feasible solutions getting closer to an \emph{optimal
    solution}.
  \pause
    \begin{halfshyblock}{Back to Elections}
        Give a feasible solution with as small objective value as possible.
    \end{halfshyblock}
    \pause
    \begin{alertblock}{Answer}
      You might have hard time guessing the right answer : $27.927927927927\ldots$
    \end{alertblock}
\end{frame}

\begin{frame}{Standard Form}
  Here are two examples apparently different from the dummy linear
  program we started with:
    \begin{overlayarea}{\textwidth}{.5\textheight}
    \begin{onlyenv}<2>
    \begin{figure}
        \begin{linearProg}{
            minimize
            }{
            $x_1 + x_2 + x_3 + x_4$
            }{
            \systeme{2x_1 - 8x_2 - 0x_3 - 10x_4 \leq -50,
                      5x_1 + 2x_2 + 0x_3 + 0x_4 \geq 100,
                      3x_1 - 5x_2 + 10x_3 - 2x_4 \geq 25}
            }{
            $x_1$, $x_2$, $x_3$, $x_4 \geq 0$
            }
        \end{linearProg}
    \end{figure}
    \end{onlyenv}
    \begin{onlyenv}<3-4>
        \vspace{-1em}
    \begin{figure}
        \begin{linearProg}{
            maximize
            }{
            \textcolor{orange}{$-x_1 - x_2 - x_3 - x_4$}
            }{
            \systeme{-2x_1 + 8x_2 + 0x_3 + 10x_4 \geq 50,
                      5x_1 + 2x_2 + 0x_3 + 0x_4\geq 100,
                      3x_1 - 5x_2 + 10x_3 - 2x_4 \geq 25}
            }{
            $x_1$, $x_2$, $x_3$, $x_4 \geq 0$
            }
        \end{linearProg}
    \end{figure}
    \end{onlyenv}
    \end{overlayarea}
    \pause[4]
    The first linear program is obtained out of the dummy case by
    multiplying the first linear inequality by $-1$. The second one by
    choosing to maximize the opposite of the objective function rather
    than minimizing it.
\end{frame}

\begin{frame}{Standard Form}
  If our aim is to build an algorithm enabling us to solve linear
  programs, such an algorithm will take as input the list of
  coefficients of our linear program. The previous examples, giving
  different coefficients for the apparently same linear program,
  suggests that we should fix a standard input form.
    \begin{defn}
      A linear program is in \emph{standard form} if
      \begin{itemize}
      \item[\textbullet] it is a maximization problem
      \item[\textbullet] all constraints are \textit{less-or-equal-to}
        inequalities
      \item[\textbullet] variables are subject to a nonnegativity
        constraint.
      \end{itemize}
    \end{defn}
    Out of the three forms our dummy program can take, none is in
    standard form.
\end{frame}

\begin{frame}{Getting a Program into Standard Form}
  A first \textit{sanity check} is to check that \emph{every} linear
  program can be written in standard form. It means that, in a similar
  fashion as to what happens for the Gauss elimination algorithm, any
  linear program should be \textit{equivalent} to a linear program
  which is in standard form.

  The notion of equivalence of two linear programs is more involved
  than in the simple case of linear systems. We shall take our time
  explaining it.
\end{frame}

\begin{frame}{Getting a Program into Standard Form}
  \begin{defn}
    \begin{itemize}
    \item[\textbullet]
      Two maximization linear programs $L$ and $L'$ are
      \emph{equivalent} if for each feasible solution of $L$
      having objective value $v$ there is a feasible solution of
      $L'$ having same objective value, and conversely
    \item[\textbullet]
      A minimization linear program $L$ is \emph{equivalent} to
      maximization linear program $L'$ if for each feasible
      solution $x$ of $L$ having objective value $v$ there is a
      feasible solution $x'$ of $L'$ having objective value $-v$,
      and conversely.
    \end{itemize}
  \end{defn}
  \begin{overlayarea}{\textwidth}{.3\textheight}
    \vspace{-1em}
    \begin{onlyenv}<2>
      \begin{prop}
        If $L$ and $L'$ are equivalent maximization linear programs
        then $L$ and $L'$ have same optimal objective values.
      \end{prop}
    \end{onlyenv}
    \begin{onlyenv}<3>
      \vspace{1em}
      \begin{halfshyblock}{Proof}
        Try justifying with your own words the fact two equivalent
        maximization linear programs have same optimal solutions.
      \end{halfshyblock}
    \end{onlyenv}
    \begin{onlyenv}<4->
      \vspace{1em}
      \begin{halfshyblock}{Variants}
        How would the statement adapt to the case of minimization
        linear programs? A maximization and a minimization one?
      \end{halfshyblock}
    \end{onlyenv}
  \end{overlayarea}
\end{frame}

\begin{frame}
  \frametitle{Examples of Equivalence Relations}
  \begin{columns}
    \begin{column}{.55\textwidth}
      \begin{figure}
        \begin{linearProg}{
            maximize
          }{
            $2x_1 + x_2 + 3$
          }{
            \systeme{x_1 + x_2 + x_3 \geq 4,
              x_1 - x_2 - x_3\geq 2,
              - x_1 + x_2 + 2x_3 \geq 1}
          }{
            $x_1$, $x_2$, $x_3 \geq 0$
          }
        \end{linearProg}
      \end{figure}
    \end{column}
    \vrule{}

    \quad
    \begin{column}{.55\textwidth}
      \begin{overlayarea}{\textwidth}{.55\textheight}

        \begin{onlyenv}<2>
         ~
          \begin{figure}
            \begin{linearProg}{
                minimize
              }{
                $-2x_1 - x_2 - 3$
              }{
                \systeme{x_1 + x_2 + x_3\phantom{-s} \geq 4,
                  x_1 - x_2 - x_3\geq 2,
                  - x_1 + x_2 + 2x_3 \geq 1}
              }{
                $x_1$, $x_2$, $x_3 \geq 0$
              }
            \end{linearProg}
          \end{figure}
        \end{onlyenv}
        \begin{onlyenv}<3>
          \begin{figure}
            \begin{linearProg}{
                minimize
              }{
                $-2x_1 - x_2 - 3$
              }{
                \systeme{x_1 + x_2 + x_3 - s = 4,
                  x_1 - x_2 - x_3\geq 2,
                  - x_1 + x_2 + 2x_3 \geq 1}
              }{
                $x_1$, $x_2$, $x_3$, $s \geq 0$
              }
            \end{linearProg}
          \end{figure}
        \end{onlyenv}
        \begin{onlyenv}<4>
          \begin{figure}
            \begin{linearProg}{
                maximize
              }{
                $2x_1 + x_2 + 3$
              }{
                \systeme{x_1 + x_2 + x_3 - s = 4,
                  x_1 - x_2 - x_3\geq 2,
                  - x_1 + x_2 + 2x_3 \geq 1}
              }{
                $x_1$, $x_2$, $x_3$, $s \geq 0$
              }
            \end{linearProg}
          \end{figure}
        \end{onlyenv}
        \begin{onlyenv}<5>
          \begin{figure}
            \begin{linearProg}{
                maximize
              }{
                $t$
              }{
                \systeme{
                  2x_1 + x_2 + 3 - t \geq 0,
                  x_1 + x_2 + x_3 \geq 4,
                  x_1 - x_2 - x_3\geq 2,
                  - x_1 + x_2 + 2x_3 \geq 1}
              }{
                $x_1$, $x_2$, $x_3$, $t \geq 0$
              }
            \end{linearProg}
          \end{figure}
        \end{onlyenv}
      \end{overlayarea}
    \end{column}
  \end{columns}
\end{frame}

\begin{frame}{Getting a Program into Standard Form}
  A linear program might not be in standard form for any of the four
  reasons hereafter :
  \begin{enumerate}
  \item The objective function might be a minimization program
    rather than a maximization.
  \item There might be variables without nonnegativity constraints.
  \item There might be \textit{equality constraints} rather than a
    less-or-equal-to signs.
  \item There might be \textit{inequality constraints} that are
    greater-or-equal-to signs rather than less-or-equal-to signs.
  \end{enumerate}
  \vspace{1em}
  \begin{overlayarea}{\textwidth}{.3\textheight}
    \begin{onlyenv}<2>
      \begin{halfshyblock}{How-do?}
        Given linear programs having one of these degeneracies at a
        time, how would you get equivalent programs in standard
        form? What if a linear program has many of these?
      \end{halfshyblock}
    \end{onlyenv}
    \begin{onlyenv}<3->
      \begin{halfshyblock}{Dummy case}
        Put our dummy case into standard form\footnote{We are abusing
          terminology : find an equivalent linear program which is in
          standard form}. What operations did you allow yourself? Why
        do they get you an equivalent program to the one you start
        with?
      \end{halfshyblock}
    \end{onlyenv}
  \end{overlayarea}
\end{frame}

\begin{frame}{Slack Form}
  Though standard form comes naturally, to work out the \emph{simplex
    algorithm} we choose to use another equivalent form : the slack
  form of a linear program.
  \begin{defn}
    Given a linear program in standard form $L$ the slack form of $L$
    is the equivalent linear program obtained out of $L$ by inserting
    an extra (non-negative) \emph{slack} variable on the LHS of each
    constraint, which is not a non-negativity one. Each inequality is
    then replaced by an equality.
  \end{defn}
  \vspace{1em}
  \begin{overlayarea}{\textwidth}{.35\textheight}
    \begin{onlyenv}<2>
      For instance the two inequalities
      \[
      \sysdelim..\systeme{x_1 + x_2 \leq 20, 2x_1 - x_2 \leq 2}
      \]
    \end{onlyenv}
    \begin{onlyenv}<3-4>
      give the equalities
      \[
      \sysdelim..\systeme{20 - x_1 - x_2 = x_3, 2 - 2x_1 + x_2 = x_4 }
      \]
      \pause[4] Are they equivalent?
    \end{onlyenv}
    \begin{onlyenv}<5->
      \vspace{-\baselineskip}
      \begin{prop}
        Standard and slack forms of a linear program are equivalent.
      \end{prop}
    \end{onlyenv}
  \end{overlayarea}
\end{frame}

\begin{frame}{Slack Form}
  \begin{halfshyblock}{Slacking}
    Put our dummy linear programs into slack form. Can you imagine a
    reason as to why adding such extra variables is natural?
  \end{halfshyblock}
\end{frame}

\begin{frame}
  \begin{center}
    {\huge \textbf{That's it for today!}}
   \end{center}
 \end{frame}

%%%%%%%%%%%%%%%%%%%%%%%%%%%%%%%%%%%%%%%%%%%%%%%%%%%%%%%%%%%%%%%

\end{document}

%%% Local Variables:
%%% mode: latex
%%% TeX-master: t
%%% End:
