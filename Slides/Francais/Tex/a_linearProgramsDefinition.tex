\documentclass[aspectratio = 169]{beamer}

\usepackage[utf8]{inputenc} % Character encoding.

\pdfinfo{
   /Author (Bashar Dudin)
   /Title  (What is a linear program?)
   /Subject (Linear Programs)
}

\usepackage{./Style/linearProgramsBeamer} % This is extra styling for Beamer environments.
\usepackage{./Style/linearProgramsStyle} % This is a set of commands for maths content.

%-------------------------------------------------------------------------------
%   TITLE PAGE
%-------------------------------------------------------------------------------

\author[BD]{Bashar Dudin}

\institute[]{EPITA}

\title{Programmes lin\'eaires} %
\subtitle{Qu'est-ce qu'un programme lin\'eaire?}

%-------------------------------------------------------------------------------
%   DOCUMENT BODY
%-------------------------------------------------------------------------------

\begin{document}

\begin{frame}[plain]
\titlepage % Print the title page as the first slide
\end{frame}

\begin{frame}{Consid\'erations politiciennes}
  \begin{onlyenv}<1>
    Cet exemple est extrait de \textbf{Introduction to Algorithms}
    par \emph{Cormen et Al.} :
        \begin{quotation}
          Suppose that you are a politician trying to win an
          election. Your district has three different types of areas:
          urban, suburban, and rural. These areas have respectively:
          100.000, 200.000 and 50.000 registered voters. Although not
          all the registered voters actually go to the polls, you
          decide that to govern effectively, you would like at least
          half the registered voters in each of the three regions to
          vote for you. You are honorable and would never consider
          supporting policies in which you do not believe. You
          realize, however, that certain issues may be more effective
          in winning votes in certain places. Your primary issues are
          building roads, gun control, farm subsidies, and a gasoline
          tax dedicated to improve public transit. According to your
          campaign staff's research, you can estimate how many voters
          you win or lose from each population segment by spending
          1.000€ on advertising on each issue.
        \end{quotation}
    \end{onlyenv}
    \begin{onlyenv}<2>
        \begin{figure}
            \begin{tabular}{l|ccc}
                politique & z. urbaine & p\'eriph\'erie & z. rurale \\
                \hline
                R\'eseau Routier & -2 & 5 & 3 \\
                Contr\^ole des armes & 8 & 2 & -5 \\
                Subvention de l'agriculture & 0 & 0 & 10 \\
                Taxer le diesel & 10 & 0 & -2
            \end{tabular}
        \end{figure}
        \begin{quotation}
          Dans la table ci-dessus, chaque entr\'ee indique le nombre
          en \emph{milliers} des votants des zones urbaines,
          p\'eriph\'eriques ou rurales dont on gagne le soutien en
          d\'epensant 1.000€ de publicit\'es sur une politique en
          particulier. Les entr\'ees n\'egatives indiques le nombres
          de votants qu'on perds. Notre but est de trouver le plus
          petit budget pub qui nous permet de gagner au moins 50.000
          votes urbains, 100.000 votes des p\'eriph\'eries et 25.000
          votes des zones rurales.
        \end{quotation}
    \end{onlyenv}
\end{frame}

\begin{frame}{Mod\'elisation}
  \begin{onlyenv}<1-2>
    Le co\^ut totale de la campagne est \'egal au co\^ut des campagnes
    de pubs sur chaque mesure.  On note $x_1$, $x_2$, $x_4$ and $x_5$
    les co\^uts respectifs des publicit\'es concernant le r\'eseau de
    routes, le contr\^ole des armes, les subventions agricoles et la
    taxation du diesel. Chaque tel co\^ut est positif. Ainsi
    \begin{equation}
      \label{eq:positivityCondition}
      x_1, x_2, x_3, x_4 \geqslant 0.
    \end{equation}
    La premi\`ere colonne donnne les contraintes suivantes sur chaque co\^ut
    \begin{equation}
      \alt<1>{-2x_1 + 8x_2 + 0x_3 + 10x_4 \geqslant 50}{\textcolor{orange}{-2x_1 + 8x_2 + 10x_4 \geqslant 50}}
    \end{equation}
    Le membre de gauche de l'in\'egalite d\'ecrit le fait qu'il nous
    faut plus que 50.000 votants des zones urbaines.
  \end{onlyenv}
  \begin{onlyenv}<3>
    Chaque colonne de la table des co\^uts donne une telle
    \emph{contrainte d'in\'egalit\'e} sur l'ensemble des co\^uts. En
    r\'esum'e, notre probl\`eme s'\'ecrit sous la forme
    \begin{figure}
      \begin{linearProg}{
          minimiser
        }{
          $x_1+x_2+x_3+x_4$
        }{
          \systeme{-2x_1 + 8x_2 + 0x_3 + 10x_4 \geq  50,
            5x_1 + 2x_2 + 0x_3 + 0x_4\geq 100,
            3x_1 - 5x_2 + 10x_3 - 2x_4 \geq 25}
        }{
          $x_1$, $x_2$, $x_3$, $x_4 \geq 0$
        }
      \end{linearProg}
    \end{figure}
    Ceci est notre premier exemple de probl\'eme lin\'eaire.
    \end{onlyenv}
\end{frame}

\begin{frame}{D\'efinition G\'en\'erale}
  \begin{defn}
    Un programme lin\'eaire est un probl\'eme de minimisation ou de
    maximisation d'une fonction lin\'eaire sujet \`a un nombre de
    contraintes \'egalement lin\'eaires.
  \end{defn}
  \alt<3>{Une \alert{\emph{fonction affine}}}{Une \emph{fonction
      lin\'eaire}} en $k$ variables (pour $k \in \N^*$) est une fonction
  $f$ ayant une expression de la forme
  \[
    f(x_1, x_2, \ldots, x_k) = \uncover<3>{\alert{c_0 + }} c_1x_1 + c_2x_2 + \cdots + c_kx_k
  \]
  pour $k$ nombres r\'eels $c_1$, $\ldots$, $c_k$. Une
  \emph{contrainte lin\'eaire} est ou bien une \'egalit\'e ou alors
  une in\'egalit\'e (large), contenant uniquement des expressions
  affines.

  \pause La \alt<3>{\alert{\textit{fonction affine}}}{\textit{fonction
      lin\'eaire}} qu'on cherche \`a optimiser dans un programme
  lin\'eaire est appel\'ee la \emph{fonction objectif}. Quand on
  cherche \`a minimiser la fonction objectif on dit qu'on a affaire
  \`a un probl\`eme de minimisation. C'est un probl\`eme de
  maximisation sinon.
\end{frame}

\begin{frame}{D\'efinition G\'en\'erale}
  \'Etant donn\'e un programme lin\'eaire, notre but est de trouver
  des points qui satisfonts les contraintes lin\'eaires, et pour
  lesquels la fonction objectif prend sa valeur optimale. Une valeur
  prise par la fonction objectif est appel\'ee une \emph{valeur
    objectif}.

  \pause
  Une solution aux contraintes lin\'eaires, m\^eme si elle n'optimise
  pas la fonction objectif, est dite \emph{solution
    admissible}. L'essentiel des algorithmes qui optimisent des
  probl\`emes lin\'eaires proc\`edent en construisant des suites de
  solutions admissibles it\'erativement, et qui convergent vers une
  \emph{solution optimale}.  \pause
  \begin{halfshyblock}{Retour aux \'el\'ections}
    Donner des solutions admissibles ayant d'aussi petites valeurs que
    vous pouvez.
  \end{halfshyblock}
  \pause
  \begin{alertblock}{Tip}
    La vraie r\'eponse n'est peut-\^etre pas si facile que cela.
  \end{alertblock}
\end{frame}

\begin{frame}{Forme standard}
  Voici deux exemples apparemment diff\'erent du programme lin\'eaire
  politique:
    \begin{overlayarea}{\textwidth}{.5\textheight}
    \begin{onlyenv}<2>
    \begin{figure}
        \begin{linearProg}{
            minimiser
            }{
            $x_1 + x_2 + x_3 + x_4$
            }{
            \systeme{2x_1 - 8x_2 - 0x_3 - 10x_4 \leq -50,
                      5x_1 + 2x_2 + 0x_3 + 0x_4 \geq 100,
                      3x_1 - 5x_2 + 10x_3 - 2x_4 \geq 25}
            }{
            $x_1$, $x_2$, $x_3$, $x_4 \geq 0$
            }
        \end{linearProg}
    \end{figure}
    \end{onlyenv}
    \begin{onlyenv}<3-4>
        \vspace{-1em}
    \begin{figure}
        \begin{linearProg}{
            maximiser
            }{
            \textcolor{orange}{$-x_1 - x_2 - x_3 - x_4$}
            }{
            \systeme{-2x_1 + 8x_2 + 0x_3 + 10x_4 \geq 50,
                      5x_1 + 2x_2 + 0x_3 + 0x_4\geq 100,
                      3x_1 - 5x_2 + 10x_3 - 2x_4 \geq 25}
            }{
            $x_1$, $x_2$, $x_3$, $x_4 \geq 0$
            }
        \end{linearProg}
    \end{figure}
    \end{onlyenv}
    \end{overlayarea}
    \pause[4]
    Le premier programme est obtenu \'a partir de notre exemple de
    d\'epart en multipliant la premi\`ere contrainte par $-1$. Le
    second en choisissant de maximiser l'oppos\'e de la fonction
    objectif au lieu de la minimiser.
\end{frame}

\begin{frame}{Forme standard}
  Si notre but est de construire un algorithme qui nous permette de
  r\'esoudre des programmes lin\'eaires, un tel algorithme va prendre
  en entr\'ee la liste des coefficients de notre programmes. Les
  exemples pr\'ec\'edents sugg\`erent qu'on a diff\'erentes listes de
  coefficients possibles pour un m\^eme programme lin\'eaire. Il nous
  faut donc standardiser l'entr\'ee.
  \begin{defn}
    Un programme lin\'eaire est sous \emph{forme standard} si
    \begin{itemize}
    \item[\textbullet] c'est un probl\`eme de maximisation
    \item[\textbullet] toutes les contraintes sont des in\'egalit\'es larges
    \item[\textbullet] toutes les variables sont sujet \`a des
      contraintes de positivit\'e.
    \end{itemize}
  \end{defn}
  Des trois formes que notre programme d'\'etude \` prise aucune n'est
  sous forme standard.
\end{frame}

\begin{frame}{R\'eduction \`a la forme standard}
  Afin que la notion de forme standard puisse \^etre int\'er\'essante
  il faut qu'on puisse transformer n'importe quel programme lin\'eaire
  en un programme lin\'eaire sous forme standard. On s'attend, de
  fa\c{c}on similaire \`a ce qu'on a dans le cas du pivot de Gauss,
  \`a une notion d'\emph{\'equivalence} de programmes lin\'eaires pour
  laquelle tous programmes lin\'eaire est \'equivalent \`a un
  programme sous forme standard.


  Cette notion d'\'equivalence de programme lin\'eaire est assez
  d\'elicate ; on va prendre le temps de l'expliquer dans le d\'etail.
\end{frame}

\begin{frame}{R\'eduction \`a la forme standard}
  \begin{defn}
    \begin{itemize}
    \item[\textbullet]
      Deux programmes lin\'eaires de maximisation $L$ et $L'$ sont
      \emph{equivalent} si pour toute solution admissible  de $L$
      ayant valeur objectif $v$ il existe une solution admissible de
      $L'$ ayant m\^eme valeur objectif, et vice versa.
    \item[\textbullet]
      Un programme de minimisation $L$ est \emph{equivalent} \`a un
      porgramme de maximization $L'$ si pour chaque solution
      admissible $x$ de $L$ ayant valeur objectif $v$ il existe une
      solution admissible $x'$ de $L'$ ayant valeur objectif $-v$, et
      vice versa.
    \end{itemize}
  \end{defn}
  \begin{overlayarea}{\textwidth}{.3\textheight}
    \vspace{-1em}
    \begin{onlyenv}<2>
      \begin{prop}
        Si $L$ et $L'$ sont des programmes de maximization
        \'equivalents alors $L$ et $L'$ ont m\^emes valeurs optimales.
      \end{prop}
    \end{onlyenv}
    \begin{onlyenv}<3>
      \vspace{1em}
      \begin{halfshyblock}{Proof}
        Utilisez vos mots pour justifier ce fait.
      \end{halfshyblock}
    \end{onlyenv}
    \begin{onlyenv}<4->
      \vspace{1em}
      \begin{halfshyblock}{Variantes}
        Comment s'adapte cet \'enonc\'e au cas de minimsations? Au cas mixte?
      \end{halfshyblock}
    \end{onlyenv}
  \end{overlayarea}
\end{frame}

\begin{frame}
  \frametitle{Examples de cas d'\'equivalences}
  \begin{columns}
    \begin{column}{.55\textwidth}
      \begin{figure}
        \begin{linearProg}{
            maximiser
          }{
            $2x_1 + x_2 + 3$
          }{
            \systeme{x_1 + x_2 + x_3 \geq 4,
              x_1 - x_2 - x_3\geq 2,
              - x_1 + x_2 + 2x_3 \geq 1}
          }{
            $x_1$, $x_2$, $x_3 \geq 0$
          }
        \end{linearProg}
      \end{figure}
    \end{column}
    \vrule{}

    \quad
    \begin{column}{.55\textwidth}
      \begin{overlayarea}{\textwidth}{.55\textheight}

        \begin{onlyenv}<2>
         ~
          \begin{figure}
            \begin{linearProg}{
                minimiser
              }{
                $-2x_1 - x_2 - 3$
              }{
                \systeme{x_1 + x_2 + x_3\phantom{-s} \geq 4,
                  x_1 - x_2 - x_3\geq 2,
                  - x_1 + x_2 + 2x_3 \geq 1}
              }{
                $x_1$, $x_2$, $x_3 \geq 0$
              }
            \end{linearProg}
          \end{figure}
        \end{onlyenv}
        \begin{onlyenv}<3>
          \begin{figure}
            \begin{linearProg}{
                minimiser
              }{
                $-2x_1 - x_2 - 3$
              }{
                \systeme{x_1 + x_2 + x_3 - s = 4,
                  x_1 - x_2 - x_3\geq 2,
                  - x_1 + x_2 + 2x_3 \geq 1}
              }{
                $x_1$, $x_2$, $x_3$, $s \geq 0$
              }
            \end{linearProg}
          \end{figure}
        \end{onlyenv}
        \begin{onlyenv}<4>
          \begin{figure}
            \begin{linearProg}{
                maximiser
              }{
                $2x_1 + x_2 + 3$
              }{
                \systeme{x_1 + x_2 + x_3 - s = 4,
                  x_1 - x_2 - x_3\geq 2,
                  - x_1 + x_2 + 2x_3 \geq 1}
              }{
                $x_1$, $x_2$, $x_3$, $s \geq 0$
              }
            \end{linearProg}
          \end{figure}
        \end{onlyenv}
        \begin{onlyenv}<5>
          \begin{figure}
            \begin{linearProg}{
                maximiser
              }{
                $t$
              }{
                \systeme{
                  2x_1 + x_2 + 3 - t \geq 0,
                  x_1 + x_2 + x_3 \geq 4,
                  x_1 - x_2 - x_3\geq 2,
                  - x_1 + x_2 + 2x_3 \geq 1}
              }{
                $x_1$, $x_2$, $x_3$, $t \geq 0$
              }
            \end{linearProg}
          \end{figure}
        \end{onlyenv}
      \end{overlayarea}
    \end{column}
  \end{columns}
\end{frame}

\begin{frame}{R\'eduction \`a la forme standard}
  Un programme lin\'eaire pourrait ne pas \^etre sous forme standard
  \`a cause de l'un des probl\`emes suivants:
  \begin{enumerate}
  \item La fonction objectif n'est pas \`a maximiser mais \`a
    minimiser.
  \item Une variable peut ne pas \^etre sans contraintes de positivit\'e.
  \item Il se peut qu'on ait une contrainte d'\'egalit\'e et non une
    contrainte d'in\'egalit\'e large.
  \item Il se peut que certaines contraintes d'in\'egalit\'es soient invers\'ees.
  \item Il se peut que certaines contraintes d'in\'egalit\'es ne
    soient pas larges, mais strictes.
  \end{enumerate}
  \vspace{1em}
  \begin{overlayarea}{\textwidth}{.3\textheight}
    \begin{onlyenv}<2>
      \begin{halfshyblock}{How-do?}
        Comment r\'egler chacune des pathologies pr\'ec\'edentes.
      \end{halfshyblock}
    \end{onlyenv}
    \begin{onlyenv}<3->
      \begin{halfshyblock}{Cas d'\'etude}
        Mettex notre programme lin\'eaire sous forme standard. Quelles
        sont les op\'erations que vous avex effectu\'ees? Pourquoi
        donnent-elle un programme lin\'eaire \'equivalent \`a
        l'original?
      \end{halfshyblock}
    \end{onlyenv}
  \end{overlayarea}
\end{frame}

\begin{frame}{Forme \textit{slack}}
  Malgr\'e le fait que la forme standard soit naturelle, l'algorithme
  du \emph{simplexe} on travaille avec une forme \'equivalente dite
  \emph{slack} ou des \emph{\'ecarts}.
  \begin{defn}
    \'Etand donn\'e un programme lin\'eaire sous forme standard $L$ la
    forme \textit{slack} de $L$ est obtenue \`a partir de $L$ en
    injectant une variable suppl\'ementaire (dite d'\'ecart) pour
    chaque in\'egalit\'e comme membre gauche. Chaque in\'egalit\'e est
    d\`es lors remplac\'ee par une \'egalit'e, et les variables
    d'\'ecarts subissent des contraintes d'in\'egalit\'es.
  \end{defn}
  \vspace{1em}
  \begin{overlayarea}{\textwidth}{.35\textheight}
    \begin{onlyenv}<2>
      Par exemples les deux in\'egalit\'es
      \[
      \sysdelim..\systeme{x_1 + x_2 \leq 20, 2x_1 - x_2 \leq 2}
      \]
    \end{onlyenv}
    \begin{onlyenv}<3-4>
      donnent les in\'egalit\'es
      \[
      \sysdelim..\systeme{20 - x_1 - x_2 = x_3, 2 - 2x_1 + x_2 = x_4 }
      \]
    \end{onlyenv}
  \end{overlayarea}
\end{frame}

\begin{frame}{Slack Form}
  \begin{prop}
    Les formes standard et \textit{slack} d'un programme
    lin\'eaire sont \'equivalentes.
  \end{prop}
  \pause
  \begin{halfshyblock}{Slacking}
    Donnez la forme standard de notre programme d'\'etude.
  \end{halfshyblock}
\end{frame}

\begin{frame}
  \begin{center}
    {\huge \textbf{C'est tout pour aujourd'hui!}}
   \end{center}
 \end{frame}

%%%%%%%%%%%%%%%%%%%%%%%%%%%%%%%%%%%%%%%%%%%%%%%%%%%%%%%%%%%%%%%

\end{document}

%%% Local Variables:
%%% mode: latex
%%% TeX-master: t
%%% End:
