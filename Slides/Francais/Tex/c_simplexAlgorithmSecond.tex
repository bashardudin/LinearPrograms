\documentclass[aspectratio = 169]{beamer}

\usepackage[utf8]{inputenc} % Character encoding.

\pdfinfo{
   /Author (Bashar Dudin)
   /Title  (The Simplex Algorithm II)
   /Subject (Linear Programs)
}


\usepackage{./Style/linearProgramsBeamer} % This is extra styling for Beamer environments.
\usepackage{./Style/linearProgramsStyle} % This is a set of commands for maths content.

%-------------------------------------------------------------------------------
%   TITLE PAGE
%-------------------------------------------------------------------------------

\author[BD]{Bashar Dudin}

\institute[]{EPITA}

\title{Linear Programs} %
\subtitle{The Simplex Algorithm II}

%-------------------------------------------------------------------------------
%   DOCUMENT BODY
%-------------------------------------------------------------------------------

\begin{document}

\begin{frame}[plain]
\titlepage % Print the title page as the first slide
\end{frame}

\begin{frame}{Où on en est, à quoi on fait face.}
  La procédure adoptée jusqu'à présent s'est toujours terminée jusqu'à
  présent. Notre conjecture de base est qu'elle nous renvoie un point
  optimal du programme linéaire de départ. Pour l'instant on ne peut
  pas en être certains. Par bien des égards ce qu'on a fait jusqu'à
  présent est loin d'être satisfaisants. Étant donné un programme
  linéaire $L$, voici la liste des choses auxquelles on n'a pas encore
  répondu:
    \begin{itemize}
    \item[\textcolor<6>{lightgray}{\textbullet}]<1->
      \textcolor<6>{lightgray}{on a aucun moyen de teste si $L$ est
        admissible.}
    \item[\textcolor<6>{lightgray}{\textbullet}]<2->
      \textcolor<6>{lightgray}{Comment traiter le cas où la solution
        de base n'est pas admissible?}
        \item[\textbullet]<3->
          Comment vérifier si $L$ n'est pas majoré?
        \item[\textbullet]<4-> Est-ce que la procédure qu'on a pu
          tester jusqu'à présent se termine en général?
        \item[\textcolor<6>{lightgray}{\textbullet}]<5->
          \textcolor<6>{lightgray}{Si la procédure se termine, est-ce
            qu'on a une valeur optimale?}
    \end{itemize}
\end{frame}

\section{Pivoter}

\begin{frame}{Pivoter | Fixer la notatiion}
  Dans le but de répondre aux deux questions précédentes il va nous
  falloir écrire au propre les algorithmes qu'on a testé à la main. On
  considère le programme llinéaire $L$
  \begin{figure}
    \alt<4>{
      \textcolor{orange}{
        \begin{linearProgG}{
            ${\displaystyle z - \sum_{j=1}^n c_jx_j} = \nu$
          }{
            ${\displaystyle \forall i \in \{1, \ldots, m\}, \quad \sum_{j=1}^n a_{ij}x_j + x_{i+m} = b_i}$
          }{
            $\forall j \in N\cup B, \quad x_j \geq 0$
          }
        \end{linearProgG}
      }
    }{
      \alt<3>{
        \begin{linearProgG}{
            ${\displaystyle z = \nu + \sum_{j=1}^n c_jx_j}$
          }{
            ${\displaystyle \forall i \in \{1, \ldots m \}, \quad x_{i+m} = b_i - \sum_{j=1}^n a_{ij}x_j}$
          }{
            $\forall j \in N\cup B, \quad x_j \geq 0.$
          }
        \end{linearProgG}
      }{
        \begin{linearProgG}{
            ${\displaystyle z = \nu + \sum_{j=1}^n c_jx_j}$
          }{
            ${\displaystyle \forall i \in \{1, \ldots, m\}, \quad \sum_{j=1}^n a_{ij}x_j \leq b_i}$
          }{
            $\forall j \in N, \quad x_j \geq 0$
          }
        \end{linearProgG}
      }
    }
    \end{figure}
    \begin{overlayarea}{\textwidth}{.3\textheight}
        \begin{onlyenv}<1>
          On écrit
            \begin{itemize}
            \item[\textbullet]
              $\bs{A}$ pour la matrice de taille $(m, n)$ des coéfficients
              $\big(a_{ij}\big)_{\substack{1 \leq i \leq m \\ 1 \leq j \leq n}}$
            \item[\textbullet]
              $\bs{b}$ pour le $m$-tuple $(b_1, \ldots, b_m)$
            \item[\textbullet]
              $\bs{c}$ pour le $n$-tuple $(c_1, \ldots, c_n)$.
            \end{itemize}
            Noter que $\nu$ est la valeur objective de la solution de
            base.
        \end{onlyenv}
        \begin{onlyenv}<2> Le programme linéaire $L$ dans cette forme
          est décrit par la donnée $(\bs{A}, \bs{b}, \bs{c},
          \nu)$. Le programme initial a en général $\nu = 0$.
        \end{onlyenv}
        \begin{onlyenv}<3-> Pour encoder la forme slack on inclus
          également les données $N$, $B$ des indices de variables
          hors-base et de base. Une forme slack est donc donnée par
          $(N, B, \bs{A}, \bs{b}, \bs{c}, \nu)$. L'ensemble $N$ est
          initialisé à $\{1, \ldots, n\}$ et $B$ à
          $\{n+1, \ldots, n+m\}$. \pause[4] \textcolor{orange}{Pour se
            rapprocher de l'écriture d'un système linéaire on modifie
            légérement l'écriture de la forme slack.}
        \end{onlyenv}
    \end{overlayarea}
\end{frame}

\begin{frame}
  \frametitle{Pivoter | Le tableau d'un programme linéaire}
  Les contraintes linéaires de $L$ dans la forme slack représente une
  matrice $(m, n+m)$ notée $\underline{\bs{A}}$ et obtenue comme
  concaténation de $\bs{A}$ et $\bs{I}_m$ le long des lignes de
  $\bs{A}$. Le \emph{tableau} $T$ de $L$ est obtenu en
  \begin{itemize}
  \item<2-> concatenant $\underline{\bs{A}}$ et $b^T$ le long des
    colonnes de $A$
  \item<3-> concatenant les résultat avec le vecteur
    $(-c_1, \ldots, -c_n, 0, \ldots 0, \nu)$ de taille $n+m+1$ le long
    des lignes de $\bs{A}$.
  \end{itemize}
  \pause[4]
  \begin{figure}
    \begin{tabular}{c|ccc|cccc|c|}
       & \alert{$\bs{x_1}$} & \alert{$\cdots$} & \alert{$\bs{x_n}$} & \alert{$\bs{x_{n+1}}$} & \alert{$\bs{x_{n+2}}$} & \alert{$\cdots$} & \alert{$\bs{x_{n+m}}$} &  \\
      \hline
       & $-c_1$ & $\cdots$ & $-c_n$ & $0$ & $0$ & $\cdots$ & $0$ & $\nu$ \\
      \hline
      \alert{$\bs{n+1}$} & $a_{11}$ & $\cdots$ & $a_{1n}$ & $1$ & $0$ & $\cdots$ & $0$ & $b_1$ \\
      \alert{$\bs{n+2}$}& $a_{21}$ & $\cdots$ & $a_{2n}$ & $0$ & $1$ & $\cdots$ & $0$ & $b_2$ \\
      \alert{$\vdots$}& $\vdots$ & $\ddots$ & $\vdots$ & $\vdots$ & $\vdots$ & $\ddots$ & $\vdots$ & $\vdots$ \\
      \alert{$\bs{n+m}$} & $a_{m1}$ & $\cdots$  & $a_{mn}$  & $0$ & $\cdots$ & $\cdots$ & $1$ & $b_m$
    \end{tabular}
  \end{figure}
  Le tableau $T$ est de taille $(m + 1, n + m + 1)$.
\end{frame}

\begin{frame}[fragile]{Pivoter}
  We are given input $(N, B, \bs{A}, \bs{b}, \bs{c}, \nu)$ and two
  indexes $e \in N$, $l\in B$ respectively corresponding to
  entering and leaving variables to and from the set of \textit{basic}
  variables $B$.
    \begin{columns}
        \begin{column}{.5\textwidth}
            \begin{itemize}
                \item[\textbullet]<2->
                  Express $x_e$ in terms of other variables in
                  equation $l$
                \item[\textbullet]<3->
                  Replace $x_e$ by previously obtained expression in
                  linear constraints
                \item[\textbullet]<4->
                  Replace $x_e$ by corresponding expression in the
                  value function
                \item[\textbullet]<5->
                  Update basic and none basic sets of variables.
            \end{itemize}
        \end{column}
        \begin{column}{.5\textwidth}
            \begin{overlayarea}{.96\textwidth}{.45\textheight}
            \begin{onlyenv}<2->
                \begin{tcolorbox}[
                    enhanced,
                    parbox = false,
                    colback = mLightBrown!10!white,
                    colframe = mLightBrown,
                    arc = 0mm,
                    ]
                    \small{
                    \mint{python}{T[l, :] = (1/T[l, e])*T[l, :]}
                    }
                    \end{tcolorbox}
            \end{onlyenv}
            \begin{onlyenv}<3->
                \begin{tcolorbox}[
                    enhanced,
                    parbox = false,
                    colback = mLightBrown!10!white,
                    colframe = mLightBrown,
                    arc = 0mm,
                    ]
                    \small{
                      \begin{minted}[autogobble]{python}
                        if i != l:
                          T[i, :] -= T[i, e]*T[l, :]
                      \end{minted}
                    }
                \end{tcolorbox}
            \end{onlyenv}
            \begin{onlyenv}<5>
                \begin{tcolorbox}[
                    enhanced,
                    parbox = false,
                    colback = mLightBrown!10!white,
                    colframe = mLightBrown,
                    arc = 0mm,
                    ]
                    \small{
                      \begin{minted}[autogobble]{python}
                        N.insert(N.index(e), l).remove(e)
                        B.insert(B.index(l), e).remove(l)
                      \end{minted}
                    }
            \end{tcolorbox}
            \end{onlyenv}
        \end{overlayarea}
        \end{column}
    \end{columns}
\end{frame}

\begin{frame}[fragile]{Pivoter | Full Function}
     \small{
      \begin{minted}[linenos]{python}
        def pivot(N, B, T, e, l):
            """Pivoting in linear programs.

            Pivots entering and leaving variables in linear
            program given as tableau. Done in place.
            """
            T[l, :] = (1/T[l, e])*T[l, :]
            for i in range(m+1):
                if i != l:  # ugly
                    T[i, :] -= T[i, e]*T[i, :]
            N.insert(N.index(e), l).remove(e)
            B.remove(B.index(l), e).remove(l)
      \end{minted}
    }
\end{frame}

\begin{frame}{Pivoter | Full Function}
  \begin{itemize}
  \item<1->
    Index $e$ is an element of $N$ and $l$ one of $B$.
  \item<2->
    There better be no entries $e$, $l$ such that
    \mintinline{python}{T[l, e] == 0}. We shall ensure this is never
    the case when \mintinline{python}{pivot} is used.
  \item<3->
    The objective value and basic solution of obtained linear program
    can be read on last column to the right.
  \end{itemize}
\end{frame}

\section{Détecter le fait de ne pas être borné}

\begin{frame}{Tester la non \emph{bornitude}}
  On note $L$ le programme linéaire en forme standard
  \begin{figure}
    \begin{linearProgG}{
        ${\displaystyle z = \nu + \sum_{j=1}^n c_jx_j}$
      }{
        ${\displaystyle \forall i \in \{1, \ldots, m\}, \quad \sum_{j=1}^n a_{ij}x_j \leq b_i}$
      }{
        $\forall j \in N, \quad x_j \geq 0$
      }
    \end{linearProgG}
  \end{figure}
  \begin{halfshyblock}{Fait}
    \begin{onlyenv}<1
      >S'il y a un indice $j$ tel que $c_j > 0$ et
      tous les coéfficients de $x_j$ dans les contraintes linéaires
      sont négatifs alors $L$ est non majoré.
    \end{onlyenv}
    \begin{onlyenv}<2-
      >\!\alert{De manière équivalent, s'il y a une
        colonne dans le tableau de $L$ ayant une première entrée
        négative non nulle et toutes les autres négatives alors $L$
        est non majoré. if there was a column in the program's tableau
        having negative first entry and only non-positive ones
        afterwards then $L$ is unbounded.}
    \end{onlyenv}
    \end{halfshyblock}
\end{frame}

\begin{frame}{Tester la non \emph{bornitude}}
  En utilisant le fait précédent, à chaque appel de
  \mintinline{python}{pivot}, on teste si on ne satisfait pas la
  propriété:
  \begin{quote}
    Il existe un indice $j$ tel que $c_j > 0$ et tous les autres
    coefficients de $x_j$ dont négatifs ou nuls.
    \end{quote}
    \pause
    C'est une condition nécessaire, pour l'instant on a aucune
    garantie qu'on tombe dans une telle situation quand $L$ n'est pas
    majoré. Le fait que ce soit le cas vient des résultats de dualité.
    \pause
    Pour l'instant on va devoir accepter que ce test naïf sera
    suffisant.
\end{frame}

\section{The Simplex Algorithm : Second Try}

\begin{frame}[fragile]{The Simplex Algorithm \emph{Restricted}}
  \begin{overlayarea}{1.1\textwidth}{.75\textheight}
  \setlength{\columnsep}{-10pt}
  \begin{multicols}{2}
    \scriptsize{
      \begin{minted}[autogobble, linenos, breaklines]{python}
        # Under construction function!
        def _simplex(N, B, T):
            """Restricted simplex algorithm

            Runs simplex algorithm on basic feasible
            linear program in slack form.

            Args:
              N, B (list[int]): lists of non-basic
              and basic variables.
              T (ndarray[float]): numpy array for
              tableau of linear program.
            Output:
              (ndarray[float]) vector tail of which
              is maximal objective value, rest is
              optimal point.
            """
            m, margins = len(B), dict()
            aug_var = [i for i in N if T[0, i] < 0]
            while aug_var:
              e = random.choice(augmenting_var)
              for i in range(m):
                if T[i, e] > 0:
                  margins[B[i]] = T[i, -1]/T[i, e]
              if not margins:
                raise Exception("Unbounded LP")
              min_margin = min(margins.values())
              minima = [i for i in margins\
                        if margins[i] == min_margin]
              l = random.choice(minima)
              pivot(N, B, T, e, l)
              aug_var = [i for i in N if T[0, i] < 0]
            return T[:, -1]
      \end{minted}
    }
  \end{multicols}
  \end{overlayarea}
\end{frame}

% \begin{frame}{The Simplex Algorithm : First Validity Checks}
%   There are two things we need to check in order to temporarily
%   accept the previous version of the simplex algorithm
%     \begin{enumerate}
%         \item<2->
%           after each call to \mintinline{python}{pivot} the linear
%           program we get has feasible basic solution
%         \item<3->
%           \textcolor<4->{lightgray}{the objective value does not
%             decrease while looping}
%           \only<4->{\textcolor{orange}{\textbf{(We choose it that
%                 way!)}}}
%     \end{enumerate}
%     \pause[5]
%     and a last point to \textit{understand} :
%     \begin{enumerate}
%         \item[3.]
%           what does it mean for \mintinline{python}{_simplex} not to terminate? how
%           can we deal with it?
%     \end{enumerate}
% \end{frame}

% \begin{frame}[fragile]{Feasability of the basic solution after each call for \mintinline{python}{pivot}}
%     \setlength\columnseprule{.1pt}
%     \begin{multicols}{2}
%       Input is an LP $(N, B, \bs{A}, \bs{b}, \bs{c}, \nu)$ having
%       feasible basic solution, i.e. \mintinline{python}{T[1:, -1]} is
%       non-negative. We show effect of \mintinline{python}{pivot}
%       leaves an LP with feasible basic solution as well. Since
%       \mintinline{python}{pivot} is called, \mintinline{python}{T[l, e] > 0}.

%       \pause
%       Updates of $\bs{b}$ are give by the relations
%       \small{
%       \begin{minted}[autogobble]{python}
%         T[l, -1] = (1/T[l, e])*T[l, -1]
%       \end{minted}
%       }
%       and for each $i \in \{1, \ldots, m+1\}$
%       \small{
%         \begin{minted}[autogobble]{python}
%           T[i, -1] -= (T[i, e]/T[l, e])*T[i, -1]
%         \end{minted}
%       }
%       \pause
%       The only possibility for such updates to give a negative result
%       is when \mintinline{python}{T[i, e] > 0}. In that case
%       \mintinline{python}{_simplex} chooses \mintinline{python}{l} in such a
%       way that
%       \small{
%       \begin{minted}[autogobble]{python}
%         T[l, -1]/T[l, e] <= T[i, -1]/T[i, e]
%       \end{minted}
%       }
%       \pause
%       Therefore, if \mintinline{python}{-T[i, e] < 0}, mutliplying
%       previous inequality by it and adding \mintinline{python}{T[i, -1]}
%       we get that update after \mintinline{python}{pivot} is
%       non-negative.
%     \end{multicols}
% \end{frame}

\begin{frame}{Boucler}
  À chaque fois qu'on entre dans la boucle \mintinline{python}{while}
  de \mintinline{python}{_simplex} ou bien on \textit{augmente la
    valeur objective} ou alors on découvre que le PL est non-majoré. A
  priori, \mintinline{python}{_simplex} pourrait s'executer
  indéfiniment, en itérant entre des formes slack équivalentes ayant
  une valeur objective constante. On va chercher à étudier ce
  phénomène.
  \pause
    \begin{prop}[\textbf{C}]
      Soit $L$ un PL $(N, B, \bs{A}, \bs{b}, \bs{c}, \nu)$ où $\bs{A}$
      est une matrice $(m, n)$. Si la boucle de
      \mintinline{python}{_simplex} s'exécute plus de $\binom{n+m}{m}$
      fois, alors l'algorithme boucle, i.e. la boucle s'exécute
      indéfiniment en alternant entre un nombre fini de formes slack
      ayant la même valeur objective.
    \end{prop}
    \begin{rem}
      Cela signifie que dès que \mintinline{python}{_simplex} entre
      dans sa boucle principale plus $\binom{n+m}{m}$ fois alors on
      peut retourner la valeur objective et la solution de base
      courante.
    \end{rem}
    \end{frame}

    % \begin{frame}{Cycling}
    %     The proof of the above proposition is based on two facts :
    %     \begin{itemize}
    %     \item if we ever get back to a previously obtained slack form,
    %       then we are going to have the same set of options we had for
    %       equivalent slack forms once again.
    %     \item There is only a finite set of possible slack forms for
    %       the same given linear program.
    %     \end{itemize}
    %     \pause
    %     The second point is the only point we need to make clear.
    % \end{frame}

    % \begin{frame}{Cycling}
    %     \begin{lem}[$\bs{B}$]
    %       Let $L$ be an LP given by $(\bs{A}, \bs{b}, \bs{c})$. A
    %       slack form of $L$ appearing in \mintinline{python}{_simplex}
    %       is determined by the choice of a set $B$ of basic variables.
    %     \end{lem}
    %     \pause
    %     \setlength\columnseprule{.1pt}
    %     \begin{multicols}{2}
    %         \begin{demo}
    %           Such slack forms $(N, B, \bs{A}, \bs{b}, \bs{c}, \nu)$
    %           and $(N, B, \bs{A'}, \bs{b'}, \bs{c'}, \nu')$ are
    %           equivalent through the identity map.  \pause
    %           Substracting the linear constraints, we get the
    %           relations :
    %           \[
    %           0 = (\nu - \nu') + \sum_{j \in N} (c_j - c_j')x_j
    %           \]
    %           and for each $i \in B$
    %           \[
    %           0 = (b_i - b_i') - \sum_{j \in N} (a_{ij} - a_{ij}')x_j.
    %           \]
    %           We're abusing notation here ; index of coefficients is
    %           the one corresponding to line of $\bs{A}$ supporting
    %           basic variable $i$.

    %           \pause These relations being true for any vector
    %           $(x_1, \ldots, x_n)$ it is clear that for each $i \in B$
    %           and $j \in N$ we have  $\nu = \nu'$, $b_i = b_i'$, $c_j = c_j'$,
    %           $a_{ij} = a_{ij}'$.
    %         \end{demo}
    %     \end{multicols}
    % \end{frame}

\begin{frame}{Boucler}
  Rajouter un compteur de boucle à l'algo du
  \mintinline{python}{_simplex} garanti la terminaison de
  \mintinline{python}{_simplex}.
  \pause
  \begin{rem}
    Cette solution n'est pas intélligente. Il y a plusieurs solutions
    à ce problème en pratique. Dans notre cas on implémente la règle
    de \emph{Bland}: à chaque choix d'indice aux lignes $21$ et $30$
    de \mintinline{python}{_simplex}, on choisit le plus petit
    possible.
  \end{rem}
\end{frame}

\begin{frame}[fragile]{The Simplex Algorithm \emph{Restricted} | No Cycling Version}
  \begin{overlayarea}{1.1\textwidth}{.75\textheight}
  \setlength{\columnsep}{-10pt}
  \begin{multicols}{2}
    \scriptsize{
      \begin{minted}[autogobble, linenos, breaklines]{python}
        def _simplex(N, B, T):
            """Restricted simplex algorithm

            Runs simplex algorithm on basic feasible
            linear program in slack form.

            Args:
              N, B (list[int]): lists of non-basic
              and basic variables.
              T (ndarray[float]): numpy array for
              tableau of linear program.
            Output:
              (ndarray[float]) vector tail of which
              is maximal objective value, rest is
              optimal point.
            """
            m = len(B)
            l, margin = None, float('inf')
            aug_var = [i for i in N if T[0, i] < 0]
            while aug_var:
              e = min(augmenting_var)
              for i in range(m):
                if T[i, e] > 0:
                   if T[i, -1]/T[i, e] < margin:
                     margin = T[i, -1]/T[i,e]
                     l = i
              if not l:
                raise Exception("Unbounded LP")
              pivot(N, B, T, e, l)
              aug_var = [i for i in N if T[0, i] < 0]
            return T[:, -1]
      \end{minted}
    }
  \end{multicols}
  \end{overlayarea}
\end{frame}

\begin{frame}
  \centering
  {\huge \textbf{C'est tout pour aujourd'hui!}}
\end{frame}

\end{document}

%%% Local Variables:
%%% mode: latex
%%% TeX-master: t
%%% End:
