\documentclass[aspectratio = 169]{beamer}

\usepackage[utf8]{inputenc} % Character encoding.

\pdfinfo{
   /Author (Bashar Dudin)
   /Title  (The Simplex Algorithm II)
   /Subject (Linear Programs)
}


\usepackage{./Style/linearProgramsBeamer} % This is extra styling for Beamer environments.
\usepackage{./Style/linearProgramsStyle} % This is a set of commands for maths content.

%-------------------------------------------------------------------------------
%   TITLE PAGE
%-------------------------------------------------------------------------------

\author[BD]{Bashar Dudin}

\institute[]{EPITA}

\title{Linear Programs} %
\subtitle{The Simplex Algorithm II}

%-------------------------------------------------------------------------------
%   DOCUMENT BODY
%-------------------------------------------------------------------------------

\begin{document}

\begin{frame}[plain]
\titlepage % Print the title page as the first slide
\end{frame}

\begin{frame}{Where We Stand, What We Face}
  Working out a vague procedure we've managed to find supposedly
  optimal solutions for simple examples of linear programs (solutions
  were optimal, but we have no mean to know that for a fact).  This
  heuristic is far from being satisfactory though : given a linear
  program $L$, here is what we're still missing :
    \begin{itemize}
        \item[\textcolor<6>{lightgray}{\textbullet}]<1->
          \textcolor<6>{lightgray}{We have no way of testing if $L$ is
            feasible, i.e. if it has at least a feasible solution}
        \item[\textcolor<6>{lightgray}{\textbullet}]<2->
          \textcolor<6>{lightgray}{How can we tackle the case when the
            basic solution is not feasible?}
        \item[\textbullet]<3->
          How can we check if $L$ is unbounded?
        \item[\textbullet]<4->
          Does the procedure we worked out even terminate in general?
        \item[\textcolor<6>{lightgray}{\textbullet}]<5->
          \textcolor<6>{lightgray}{Do we get an optimal objective
            value if procedure terminates?}
    \end{itemize}
\end{frame}

\section{Pivoting}

\begin{frame}{Pivoting | Fixing Notation}
  In order to answer both previous questions we'll need to properly
  write down involved algorithms. Consider the linear program $L$
  \begin{figure}
    \alt<4>{
      \textcolor{orange}{
        \begin{linearProgG}{
            ${\displaystyle z - \sum_{j=1}^n c_jx_j} = \nu$
          }{
            ${\displaystyle \forall i \in \{1, \ldots, m\}, \quad \sum_{j=1}^n a_{ij}x_j + x_{i+m} = b_i}$
          }{
            $\forall j \in N\cup B, \quad x_j \geq 0$
          }
        \end{linearProgG}
      }
    }{
      \alt<3>{
        \begin{linearProgG}{
            ${\displaystyle z = \nu + \sum_{j=1}^n c_jx_j}$
          }{
            ${\displaystyle \forall i \in \{1, \ldots m \}, \quad x_{i+m} = b_i - \sum_{j=1}^n a_{ij}x_j}$
          }{
            $\forall j \in N\cup B, \quad x_j \geq 0.$
          }
        \end{linearProgG}
      }{
        \begin{linearProgG}{
            ${\displaystyle z = \nu + \sum_{j=1}^n c_jx_j}$
          }{
            ${\displaystyle \forall i \in \{1, \ldots, m\}, \quad \sum_{j=1}^n a_{ij}x_j \leq b_i}$
          }{
            $\forall j \in N, \quad x_j \geq 0$
          }
        \end{linearProgG}
      }
    }
    \end{figure}
    \begin{overlayarea}{\textwidth}{.3\textheight}
        \begin{onlyenv}<1>
          We write
            \begin{itemize}
            \item[\textbullet]
              $\bs{A}$ for the $(m, n)$ matrix of coefficients
              $\big(a_{ij}\big)_{\substack{1 \leq i \leq m \\ 1 \leq j \leq n}}$
            \item[\textbullet]
              $\bs{b}$ for the $m$-tuple $(b_1, \ldots, b_m)$
            \item[\textbullet]
              $\bs{c}$ for the $n$-tuple $(c_1, \ldots, c_n)$.
            \end{itemize}
            Notice that $\nu$ is the objective value of the basic
            solution.
        \end{onlyenv}
        \begin{onlyenv}<2>
          The linear program $L$ in its standard form is determined by
          the data $(\bs{A}, \bs{b}, \bs{c}, \nu)$. Initial program
          has $\nu = 0$ in general.
        \end{onlyenv}
        \begin{onlyenv}<3->
          To encode the slack form we include the data $N$, $B$ of
          non-basic and basic sets. A slack form is thus given by the
          data $(N, B, \bs{A}, \bs{b}, \bs{c}, \nu)$. The set $N$ is
          initialized at $\{1, \ldots, n\}$ and $B$ to
          $\{n+1, \ldots, n+m\}$. \pause[4] \textcolor{orange}{To get
            closer to a standard way of writing a linear system, we
            slightly modify presentation of slack form.}
        \end{onlyenv}
    \end{overlayarea}
\end{frame}

\begin{frame}
  \frametitle{Pivoting | Tableau of a Linear Program}
  Linear constraints of $L$ in slack form have an $(m, n+m)$ matrix
  $\underline{\bs{A}}$ given by concatenating $\bs{A}$ and $\bs{I}_m$
  along rows of $\bs{A}$. The \emph{tableau} $T$ of linear program $L$
  is obtained by
  \begin{itemize}
  \item<2-> concatenating $\underline{\bs{A}}$ and $b^T$ column-wise
    along $A$
  \item<3-> concatenate result with the vector
    $(-c_1, \ldots, -c_n, 0, \ldots 0, \nu)$ of size $n+m+1$ row-wise
    along $\bs{A}$.
  \end{itemize}
  \pause[4]
  \begin{figure}
    \begin{tabular}{c|ccc|cccc|c|}
       & \alert{$\bs{x_1}$} & \alert{$\cdots$} & \alert{$\bs{x_n}$} & \alert{$\bs{x_{n+1}}$} & \alert{$\bs{x_{n+2}}$} & \alert{$\cdots$} & \alert{$\bs{x_{n+m}}$} &  \\
      \hline
       & $-c_1$ & $\cdots$ & $-c_n$ & $0$ & $0$ & $\cdots$ & $0$ & $\nu$ \\
      \hline
      \alert{$\bs{n+1}$} & $a_{11}$ & $\cdots$ & $a_{1n}$ & $1$ & $0$ & $\cdots$ & $0$ & $b_1$ \\
      \alert{$\bs{n+2}$}& $a_{21}$ & $\cdots$ & $a_{2n}$ & $0$ & $1$ & $\cdots$ & $0$ & $b_2$ \\
      \alert{$\vdots$}& $\vdots$ & $\ddots$ & $\vdots$ & $\vdots$ & $\vdots$ & $\ddots$ & $\vdots$ & $\vdots$ \\
      \alert{$\bs{n+m}$} & $a_{m1}$ & $\cdots$  & $a_{mn}$  & $0$ & $\cdots$ & $\cdots$ & $1$ & $b_m$
    \end{tabular}
  \end{figure}
  The tableau $T$ is of shape $(m + 1, n + m + 1)$.
  \end{frame}

\begin{frame}[fragile]{Pivoting}
  We are given input $(N, B, \bs{A}, \bs{b}, \bs{c}, \nu)$ and two
  indexes $e \in N$, $l\in B$ respectively corresponding to
  entering and leaving variables to and from the set of \textit{basic}
  variables $B$.
    \begin{columns}
        \begin{column}{.5\textwidth}
            \begin{itemize}
                \item[\textbullet]<2->
                  Express $x_e$ in terms of other variables in
                  equation $l$
                \item[\textbullet]<3->
                  Replace $x_e$ by previously obtained expression in
                  linear constraints
                \item[\textbullet]<4->
                  Replace $x_e$ by corresponding expression in the
                  value function
                \item[\textbullet]<5->
                  Update basic and none basic sets of variables.
            \end{itemize}
        \end{column}
        \begin{column}{.5\textwidth}
            \begin{overlayarea}{.96\textwidth}{.45\textheight}
            \begin{onlyenv}<2->
                \begin{tcolorbox}[
                    enhanced,
                    parbox = false,
                    colback = mLightBrown!10!white,
                    colframe = mLightBrown,
                    arc = 0mm,
                    ]
                    \small{
                    \mint{python}{T[l, :] = (1/T[l, e])*T[l, :]}
                    }
                    \end{tcolorbox}
            \end{onlyenv}
            \begin{onlyenv}<3->
                \begin{tcolorbox}[
                    enhanced,
                    parbox = false,
                    colback = mLightBrown!10!white,
                    colframe = mLightBrown,
                    arc = 0mm,
                    ]
                    \small{
                      \begin{minted}[autogobble]{python}
                        if i != l:
                          T[i, :] -= T[i, e]*T[l, :]
                      \end{minted}
                    }
                \end{tcolorbox}
            \end{onlyenv}
            \begin{onlyenv}<5>
                \begin{tcolorbox}[
                    enhanced,
                    parbox = false,
                    colback = mLightBrown!10!white,
                    colframe = mLightBrown,
                    arc = 0mm,
                    ]
                    \small{
                      \begin{minted}[autogobble]{python}
                        N.insert(N.index(e), l).remove(e)
                        B.insert(B.index(l), e).remove(l)
                      \end{minted}
                    }
            \end{tcolorbox}
            \end{onlyenv}
        \end{overlayarea}
        \end{column}
    \end{columns}
\end{frame}

\begin{frame}[fragile]{Pivoting | Full Function}
     \small{
      \begin{minted}[linenos]{python}
        def pivot(N, B, T, e, l):
            """Pivoting in linear programs.

            Pivots entering and leaving variables in linear
            program given as tableau. Done in place.
            """
            T[l, :] = (1/T[l, e])*T[l, :]
            for i in range(m+1):
                if i != l:  # ugly
                    T[i, :] -= T[i, e]*T[i, :]
            N.insert(N.index(e), l).remove(e)
            B.remove(B.index(l), e).remove(l)
      \end{minted}
    }
\end{frame}

\begin{frame}{Pivoting | Full Function}
  \begin{itemize}
  \item<1->
    Index $e$ is an element of $N$ and $l$ one of $B$.
  \item<2->
    There better be no entries $e$, $l$ such that
    \mintinline{python}{T[l, e] == 0}. We shall ensure this is never
    the case when \mintinline{python}{pivot} is used.
  \item<3->
    The objective value and basic solution of obtained linear program
    can be read on last column to the right.
  \end{itemize}
\end{frame}

\section{Facing Unboundedness}

\begin{frame}{Testing Boundedness}
    Let $L$ be the following linear program in standard form
    \begin{figure}
        \begin{linearProgG}{
            ${\displaystyle z = \nu + \sum_{j=1}^n c_jx_j}$
            }{
            ${\displaystyle \forall i \in \{1, \ldots, m\}, \quad \sum_{j=1}^n a_{ij}x_j \leq b_i}$
            }{
            $\forall j \in N, \quad x_j \geq 0$
            }
        \end{linearProgG}
    \end{figure}
    \begin{halfshyblock}{Fact}
      \begin{onlyenv}<1
        >If there was an index $j$ such that $c_j > 0$ and all
        coefficients of $x_{j}$ in the linear constraints were
        non-positive then $L$ is unbounded.
      \end{onlyenv}
      \begin{onlyenv}<2-
        >\!\alert{Equivalently, if there was a column in the program's tableau
        having negative first entry and only non-positive ones
        afterwards then $L$ is unbounded.}
      \end{onlyenv}
    \end{halfshyblock}
\end{frame}

\begin{frame}{Testing Boundedness}
  Using the previous fact we can test, each time we call
  \mintinline{python}{pivot}, that the input linear program
  \alert{\emph{doesn't}} satisfy the property :
    \begin{quote}
      There is an index $j$ such that $c_j > 0$ and all coefficients
      of $x_j$ in the linear constraints are non-positive.
    \end{quote}
    \pause
    This is a necessary healthy check, but there is yet no garantee
    that this condition is fullfilled when the linear program we work
    with is unbounded. We will not have the needed machinery to
    properly answer this question untill we introduce some
    duality.

    \pause
    For the time being we'll have to accept that the naive
    check at hand will be enough.
\end{frame}

\section{The Simplex Algorithm : Second Try}

\begin{frame}[fragile]{The Simplex Algorithm \emph{Restricted}}
  \begin{overlayarea}{1.1\textwidth}{.75\textheight}
  \setlength{\columnsep}{-10pt}
  \begin{multicols}{2}
    \scriptsize{
      \begin{minted}[autogobble, linenos, breaklines]{python}
        # Under construction function!
        def _simplex(N, B, T):
            """Restricted simplex algorithm

            Runs simplex algorithm on basic feasible
            linear program in slack form.

            Args:
              N, B (list[int]): lists of non-basic
              and basic variables.
              T (ndarray[float]): numpy array for
              tableau of linear program.
            Output:
              (ndarray[float]) vector tail of which
              is maximal objective value, rest is
              optimal point.
            """
            m, margins = len(B), dict()
            aug_var = [i for i in N if T[0, i] < 0]
            while aug_var:
              e = random.choice(augmenting_var)
              for i in range(m):
                if T[i, e] > 0:
                  margins[B[i]] = T[i, -1]/T[i, e]
              if not margins:
                raise Exception("Unbounded LP")
              min_margin = min(margins.values())
              minima = [i for i in margins\
                        if margins[i] == min_margin]
              l = random.choice(minima)
              pivot(N, B, T, e, l)
              aug_var = [i for i in N if T[0, i] < 0]
            return T[:, -1]
      \end{minted}
    }
  \end{multicols}
  \end{overlayarea}
\end{frame}

\begin{frame}{The Simplex Algorithm : First Validity Checks}
  There are two things we need to check in order to temporarily
  accept the previous version of the simplex algorithm
    \begin{enumerate}
        \item<2->
          after each call to \mintinline{python}{pivot} the linear
          program we get has feasible basic solution
        \item<3->
          \textcolor<4->{lightgray}{the objective value does not
            decrease while looping}
          \only<4->{\textcolor{orange}{\textbf{(We choose it that
                way!)}}}
    \end{enumerate}
    \pause[5]
    and a last point to \textit{understand} :
    \begin{enumerate}
        \item[3.]
          what does it mean for \mintinline{python}{_simplex} not to terminate? how
          can we deal with it?
    \end{enumerate}
\end{frame}

\begin{frame}[fragile]{Feasability of the basic solution after each call for \mintinline{python}{pivot}}
    \setlength\columnseprule{.1pt}
    \begin{multicols}{2}
      Input is an LP $(N, B, \bs{A}, \bs{b}, \bs{c}, \nu)$ having
      feasible basic solution, i.e. \mintinline{python}{T[1:, -1]} is
      non-negative. We show effect of \mintinline{python}{pivot}
      leaves an LP with feasible basic solution as well. Since
      \mintinline{python}{pivot} is called, \mintinline{python}{T[l, e] > 0}.

      \pause
      Updates of $\bs{b}$ are give by the relations
      \small{
      \begin{minted}[autogobble]{python}
        T[l, -1] = (1/T[l, e])*T[l, -1]
      \end{minted}
      }
      and for each $i \in \{1, \ldots, m+1\}$
      \small{
        \begin{minted}[autogobble]{python}
          T[i, -1] -= (T[i, e]/T[l, e])*T[i, -1]
        \end{minted}
      }
      \pause
      The only possibility for such updates to give a negative result
      is when \mintinline{python}{T[i, e] > 0}. In that case
      \mintinline{python}{_simplex} chooses \mintinline{python}{l} in such a
      way that
      \small{
      \begin{minted}[autogobble]{python}
        T[l, -1]/T[l, e] <= T[i, -1]/T[i, e]
      \end{minted}
      }
      \pause
      Therefore, if \mintinline{python}{-T[i, e] < 0}, mutliplying
      previous inequality by it and adding \mintinline{python}{T[i, -1]}
      we get that update after \mintinline{python}{pivot} is
      non-negative.
    \end{multicols}
\end{frame}

\begin{frame}{Cycling}
  Each time we step into the \mintinline{python}{while} loop of the
  \mintinline{python}{_simplex} algorithm we either \textit{increase or keep
    constant} the objective value or discover the LP is
  \textit{unbounded}. A priori, \mintinline{python}{_simplex} might run
  on indefinitely among equivalent slack forms without ever increasing
  the objective value. We are going to check such behaviour can be
  detected.
  \pause
    \begin{prop}[\textbf{C}]
      Let $L$ be an LP $(N, B, \bs{A}, \bs{b}, \bs{c}, \nu)$ where
      $\bs{A}$ is an $(m, n)$ matrix. If \mintinline{python}{_simplex}
      runs more than $\binom{n+m}{m}$ iterations it does cycle,
      i.e. it wanders indefinitely along the same finite set of slack
      forms with same given objective value.
    \end{prop}
    \begin{rem}
      This means that whenever \mintinline{python}{_simplex} runs more
      than $\binom{n+m}{m}$ times then one can return any current
      objective value and basic solution.
    \end{rem}
    \end{frame}

    \begin{frame}{Cycling}
        The proof of the above proposition is based on two facts :
        \begin{itemize}
        \item if we ever get back to a previously obtained slack form,
          then we are going to have the same set of options we had for
          equivalent slack forms once again.
        \item There is only a finite set of possible slack forms for
          the same given linear program.
        \end{itemize}
        \pause
        The second point is the only point we need to make clear.
    \end{frame}

    \begin{frame}{Cycling}
        \begin{lem}[$\bs{B}$]
          Let $L$ be an LP given by $(\bs{A}, \bs{b}, \bs{c})$. A
          slack form of $L$ appearing in \mintinline{python}{_simplex}
          is determined by the choice of a set $B$ of basic variables.
        \end{lem}
        \pause
        \setlength\columnseprule{.1pt}
        \begin{multicols}{2}
            \begin{demo}
              Such slack forms $(N, B, \bs{A}, \bs{b}, \bs{c}, \nu)$
              and $(N, B, \bs{A'}, \bs{b'}, \bs{c'}, \nu')$ are
              equivalent through the identity map.  \pause
              Substracting the linear constraints, we get the
              relations :
              \[
              0 = (\nu - \nu') + \sum_{j \in N} (c_j - c_j')x_j
              \]
              and for each $i \in B$
              \[
              0 = (b_i - b_i') - \sum_{j \in N} (a_{ij} - a_{ij}')x_j.
              \]
              We're abusing notation here ; index of coefficients is
              the one corresponding to line of $\bs{A}$ supporting
              basic variable $i$.

              \pause These relations being true for any vector
              $(x_1, \ldots, x_n)$ it is clear that for each $i \in B$
              and $j \in N$ we have  $\nu = \nu'$, $b_i = b_i'$, $c_j = c_j'$,
              $a_{ij} = a_{ij}'$.
            \end{demo}
        \end{multicols}
    \end{frame}

\begin{frame}{Cycling}
  \begin{prop}[\textbf{C}]
    Let $L$ be an LP $(N, B, \bs{A}, \bs{b}, \bs{c}, \nu)$ where
    $\bs{A}$ is an $(m, n)$ matrix. If \mintinline{python}{_simplex}
    runs more than $\binom{n+m}{n}$ iterations it does cycle, i.e. it
    wanders indefinitely along the same finite set of slack forms with
    same given objective value.
  \end{prop}
  \begin{demo}
    By lemma B, there are at most as many different slack forms
    as the number of possible choices of basic sets of
    variables. There are $\binom{m+n}{m}$ such choices. Thus if
    \textsc{Simplex} runs more iterations than $\binom{m+n}{m}$
    then we already obtained twice the same slack form.
  \end{demo}
  \pause
  Adding a counter to \mintinline{python}{_simplex} insures
  \mintinline{python}{_simplex} terminates.

  \pause
  \begin{rem}
    This solution is \alert{\emph{not}} an intelligent one. There are
    many different solutions in practice.  We shall implement
    \emph{Bland's rule}: each time we choose an index at line $21$ and
    $30$ of previous \mintinline{python}{_simplex}, we choose the
    smallest possible indexes.
  \end{rem}
\end{frame}

\begin{frame}[fragile]{The Simplex Algorithm \emph{Restricted} | No Cycling Version}
  \begin{overlayarea}{1.1\textwidth}{.75\textheight}
  \setlength{\columnsep}{-10pt}
  \begin{multicols}{2}
    \scriptsize{
      \begin{minted}[autogobble, linenos, breaklines]{python}
        def _simplex(N, B, T):
            """Restricted simplex algorithm

            Runs simplex algorithm on basic feasible
            linear program in slack form.

            Args:
              N, B (list[int]): lists of non-basic
              and basic variables.
              T (ndarray[float]): numpy array for
              tableau of linear program.
            Output:
              (ndarray[float]) vector tail of which
              is maximal objective value, rest is
              optimal point.
            """
            m = len(B)
            l, margin = None, float('inf')
            aug_var = [i for i in N if T[0, i] < 0]
            while aug_var:
              e = min(augmenting_var)
              for i in range(m):
                if T[i, e] > 0:
                   if T[i, -1]/T[i, e] < margin:
                     margin = T[i, -1]/T[i,e]
                     l = i
              if not l:
                raise Exception("Unbounded LP")
              pivot(N, B, T, e, l)
              aug_var = [i for i in N if T[0, i] < 0]
            return T[:, -1]
      \end{minted}
    }
  \end{multicols}
  \end{overlayarea}
\end{frame}

\begin{frame}
  \centering
  {\huge \textbf{That's it for today !}}
\end{frame}

\end{document}

%%% Local Variables:
%%% mode: latex
%%% TeX-master: t
%%% End:
