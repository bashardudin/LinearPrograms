\documentclass[32pt, aspectratio = 169]{beamer}

\usepackage[utf8]{inputenc} % Character encoding.

\pdfinfo{
   /Author (Bashar Dudin)
   /Title  (Simplex Algorithm Pre-Treatment)
   /Subject (Linear Programs)
}


\usepackage{./Style/linearProgramsBeamer} % This is extra styling for Beamer environments.
\usepackage{./Style/linearProgramsStyle} % This is a set of commands for maths content.

%-------------------------------------------------------------------------------
%   TITLE PAGE
%-------------------------------------------------------------------------------

\author[BD]{Bashar Dudin}

\institute[]{EPITA}

\title{Linear Programs} %
\subtitle{Initializing The Simplex Algorithm}

%-------------------------------------------------------------------------------
%   DOCUMENT BODY
%-------------------------------------------------------------------------------

\begin{document}

\begin{frame}[plain]
    \titlepage % Print the title page as the first slide
\end{frame}

\begin{frame}{Where We Stand, What We Face?}

  We now have an algorithm \textsc{Simplex} (\textit{Restricted}),
  which is conjectured to solve linear programs. For the time being
  we've shown that it terminates either by specifying that we have an
  unbounded linear program, or by returning a finite value
  \textit{expected} to be the maximum we are looking for.

  Uptill now, we've been working under the following two assumptions :
  \begin{itemize}
  \item
    Our linear program is \emph{feasible} ; meaning that it has at
    least one feasible solution
  \item
    The basic solution of the initial slack form is feasible.
  \end{itemize}

  \pause
  During this lecture, we are going to build a function
  \mintinline{python}{init_simplex} taking in a linear program
  $(\bs{A}, \bs{b}, \bs{c}, \nu)$ and either returning back the fact
  it is \emph{not} feasible or a linear program
  $(N, B, \underline{\bs{A}}, \underline{\bs{b}}, \underline{\bs{c}},
  \nu)$ in slack form having feasible basic solution.
\end{frame}

\section{Feasibility}

\begin{frame}{Deciding On Feasibility}
  \begin{columns}[T]
    \begin{column}{.5\textwidth}
      Let $L$ be the linear program, given in standard form:
      \begin{figure}
        \begin{linearProgG}{
            ${\displaystyle z = \nu + \sum_{j=1}^n c_jx_j}$
          }{
            ${\displaystyle \forall i \in \{1, \ldots, m\}, \quad \sum_{j=1}^n a_{ij}x_j \leq b_i}$
          }{
            $\forall j \in N, \quad x_j \geq 0$
          }
        \end{linearProgG}
      \end{figure}
    \end{column}
    \begin{column}{.58\textwidth}
      \pause
      \begin{tcolorbox}[
        enhanced,
        parbox = true,
        colback = mLightBrown!30!white,
        colframe = mLightBrown!30!white,
        arc = 0mm,
        ]
        Out of $L$ we build the following auxiliary linear program $L_{m}$:
        \vspace{-1em}
        \begin{figure}
          \begin{linearProgG}{
              ${\displaystyle z = - x_0}$
            }{
              ${\displaystyle \forall i \in \{1, \ldots, m\}, \quad \sum_{j=1}^n a_{ij}x_j - x_0\leq b_i}$
            }{
              $\forall j \in N\cup \{0\}, \quad x_j \geq 0$
            }
          \end{linearProgG}
        \end{figure}
      \end{tcolorbox}
    \end{column}
  \end{columns}
\end{frame}

\begin{frame}{Deciding On Feasibility}
    \begin{prop}
      $L$ is feasible if, and only if, the optimal objective value of
      $L_{m}$ is $0$.
    \end{prop}
    \begin{overlayarea}{\textwidth}{.8\textheight}
        \vspace{.2\baselineskip}
    \begin{onlyenv}<2-4>
        \setlength\columnseprule{.1pt}
        \begin{multicols}{2}
        \begin{demo}
          Notice first that the optimal objective value of
          $L_{m}$ is $0$. Therefore, if we show that $0$ is an
          objective value of $L_{m}$ it is necessarily the
          optimal one.\pause[3]

          If $L$ is feasible then there is a tuple
          $(t_1, \ldots, t_n)$ of non-negative real numbers satisfying
          all linear constraints of $L$. The tuple
          $(0, t_1, \ldots, t_n)$ does thus satisfy $L_{m}$ and
          $0$ is then an objective value of $L_{m}$, i.e. the
          optimal one.\pause[4]

          Conversely, if $L_{m}$ has objective value $0$ (thus
          optimal) it is of the form $(0, t_1, \ldots, t_n)$. Plugging
          this tuple in the linear constraints of $L_{m}$ it
          implies $(t_1, \ldots, t_n)$ is a solution of $L$.
        \end{demo}
    \end{multicols}
    \end{onlyenv}
    \begin{onlyenv}<5>
    The difference between $L$ and $L_{m}$ is that $L_{m}$ is always feasible :
        \begin{center}
        \begin{minipage}{.7\textwidth}
        \begin{tcolorbox}[
                enhanced,
                parbox = true,
                colback = mLightBrown!30!white,
                colframe = mLightBrown!30!white,
                arc = 0mm,
                ]
                If $b_{min}$ is the smallest negative $b_i$ for
                $i \in B$, the tuple $(-b_{min}, 0, \ldots, 0)$ is a
                feasible solution of $L_{m}$.
        \end{tcolorbox}
        \end{minipage}
        \end{center}
        Thus, temporarily admitting validity of the
        \textit{restricted} simplex algorithm, if we're able to find a
        linear program equivalent to $L_{m}$ which has \alert{feasible
          basic solution} then we can decide on the feasibilty of $L$.
    \end{onlyenv}
    \end{overlayarea}
\end{frame}

\begin{frame}{Deciding On Feasibility}
    \setlength\columnseprule{.1pt}
    \begin{multicols}{2}
        Consider the slack form of $L_{m}$
        \begin{figure}
            \begin{linearProgG}{
                ${\displaystyle z = -x_0}$
                }{
                ${\displaystyle \forall i \quad x_{i + m} + \sum_{j=1}^n a_{ij}x_j - x_0 = b_i}$
                }{
                $\forall j \in N\cup B \cup\{0 \}, \quad x_j \geq 0$
                }
            \end{linearProgG}
        \end{figure}
        \pause
        Basic solution of $L_{m}$ is not feasible as soon as $L$ is
        not, i.e. as soon as a $b_i$ is negative. \alert{\textbf{We
            assume this is the case}}.

        \pause
        Let $b_{min}$ be the minimal $b_i$ coefficient. We already
        know that
        $(-b_{min}, 0, \ldots, 0, \bs{b} - b_{min})$
        is a feasible solution of
        $L_{m}$.

        \pause
        Let's now use \mintinline{python}{pivot} with entering
        variable $0$ and leaving one $min$. The \alert{\textbf{same}}
        previous feasible solution of $L_{m}$ is now the basic
        solution of the linear program we got after pivoting.
        \vspace{.5\baselineskip}

        \pause
        \textbf{We got an equivalent linear program to $\bs{L_{m}}$
          having feasible basic solution !}
    \end{multicols}
\end{frame}

\begin{frame}
  \frametitle{Tableau of Auxiliary Linear Program}
  The tableau \mintinline{python}{T_m} of $L_m$ is obtained out of the
  one for $L$ (\mintinline{python}{T}) by adding column
  $(-1, 1, \ldots, 1)^T$ to tableau before column $\bs{b}^T$ and
  putting old coefficients of first row to $0$.
  \begin{figure}
    \begin{tabular}{c|ccc|ccccc|c|}
      & \alert{$\bs{x_1}$} & \alert{$\cdots$} & \alert{$\bs{x_n}$} & \alert{$\bs{x_{n+1}}$} & \alert{$\bs{x_{n+2}}$} & \alert{$\cdots$} & \alert{$\bs{x_{n+m}}$} & \alert{$\bs{x_0}$} &  \\
      \hline
      & $0$ & $\cdots$ & $0$ & $0$ & $0$ & $\cdots$ & $0$ & $1$ & $0$ \\
      \hline
      \alert{$\bs{n+1}$} & $a_{11}$ & $\cdots$ & $a_{1n}$ & $1$ & $0$ & $\cdots$ & $0$ & $-1$ & $b_1$ \\
      \alert{$\bs{n+2}$}& $a_{21}$ & $\cdots$ & $a_{2n}$ & $0$ & $1$ & $\cdots$ & $0$ & $-1$ & $b_2$ \\
      \alert{$\vdots$}& $\vdots$ & $\ddots$ & $\vdots$ & $\vdots$ & $\vdots$ & $\ddots$ & $\vdots$ & $\vdots$ & $\vdots$ \\
      \alert{$\bs{n+m}$} & $a_{m1}$ & $\cdots$  & $a_{mn}$  & $0$ & $\cdots$ & $\cdots$ & $1$ & $-1$ & $b_m$
    \end{tabular}
  \end{figure}
\end{frame}

\begin{frame}[fragile]{Testing Feasibility}
  \begin{overlayarea}{1.05\textwidth}{.6\textheight}
  \setlength\columnseprule{.1pt}
  \begin{multicols}{2}
    \scriptsize{
      \begin{minted}[autogobble, linenos, breaklines]{python}
        def is_feasible(N, B, T):
        """Testing feasibility of linear program.

        Args:
          N, B (list[int]): lists of non-basic and
          basic variables.
          T (ndarray[float]): numpy array for
          tableau of linear program.
        Output:
          (bool) True if program is feasible
          False otherwise.
        """
        p, q = T.shape[0], T.shape[1]
        c = T[0, :]
        T[0, :] = [0]*q
        new_c = np.array([1] + [-1]*(p-1), dtype=float)
        T = np.insert(T, -1, new_c, axis=1)
        N.append(0)
        # pivoting to be basic feasible
        i_min = np.argmin(T[1: ,-1])
        pivot(N, B, T, i_min, 0)
        V = _simplex(N, B, T)
        return V[0] == 0
      \end{minted}
    }
  \end{multicols}
\end{overlayarea}
\pause
  \begin{tcolorbox}[
    enhanced,
    parbox = false,
    colback = mLightBrown!30!white,
    colframe = mLightBrown!30!white,
    arc = 0mm,
    ]
    Not an equivalent LP to L there! Need to understand boundary effects.
  \end{tcolorbox}
\end{frame}

\section{Get A Feasible Basic Solution}

\begin{frame}{Getting a Feasible Basic Solution}
  \begin{halfshyblock}{Assumption}
    We assume \mintinline{python}{is_feasible(N, B, T)} returns
    \mintinline{python}{True}.
  \end{halfshyblock}
  \vspace{.5\baselineskip}
  \begin{overlayarea}{\textwidth}{.6\textheight}
    \begin{onlyenv}<2>
      Under the previous assumption \mintinline{python}{is_feasible}
      transforms $L_m$ into an equivalent linear program $P$ which has
      feasible basic solution with objective value $0$.
    \end{onlyenv}
    \begin{onlyenv}<3->
      \begin{itemize}
      \item<3-> Replace the objective value of $P$ with the original one of $L$.
      \item<4-> Make of it a proper LP $Q$ in slack form by replacing
        possible basic variables of the objective value with their
        expressions in terms of non-basic variables of $P$.
      \item<5-> Ensure $x_0$ is no more a basic variable $Q$ by
        pivoting with leaving variable $0$ and entering one any
        variable having non-zero coefficient.
      \end{itemize}
      \pause[6]
      \vspace{.5\baselineskip}
      \begin{halfshyblock}{Fact}
        Putting $x_0$ to $0$ in $Q$ gives back an equivalent LP to $L$
        having feasible basic solution.
      \end{halfshyblock}
    \end{onlyenv}
  \end{overlayarea}
\end{frame}

\begin{frame}{The \textsc{InitSimplex} Function}
  \small{
    \begin{algorithmic}[1]
      \Require $L = (\bs{A}, \bs{b}, \bs{c})$ a linear program in standard form
      \Ensure An equivalent program having feasible basic solution if $L$ is feasible.
      \Function{InitSimplex}{$\bs{A}, \bs{b}, \bs{c}$}
            \If{\Call{IsFeasible}{$\bs{A}, \bs{b}, \bs{c}$} raises an exception}
                \State \textbf{raise exception} \textit{infeasible}
            \Else
                \State ($N$, $B$, $\bs{A'}$, $\bs{b'}$, $\bs{c'}$, $\nu$) = \Call{IsFeasible}{$\bs{A}, \bs{b}, \bs{c}$}
                \If{$0$ is basic}
                    \State Choose $e$ such that $a_{0e}' \neq 0$
                    \State ($N$, $B$, $\bs{A'}$, $\bs{b'}$, $\bs{c'}$, $\nu$) = \Call{Pivot}{$N$, $B$, $\bs{A'}$, $\bs{b'}$, $\bs{c'}$, $\nu$, $0$, $e$}
                \EndIf
                \State Update $(\bs{c}, \nu)$ to express objective value in terms non-basic indices
                \State Remove column corresponding to index $0$ from $\bs{A'}$
                \State \Return{($N$, $B$, $\bs{A'}$, $\bs{b'}$, $\bs{c}$, $\nu$)}
            \EndIf
        \EndFunction
        \Statex
    \end{algorithmic}
    }
\end{frame}

\begin{frame}[fragile]{The \mintinline{python}{is_feasible} function}
  \begin{overlayarea}{1.05\textwidth}{.75\textheight}
    \setlength\columnseprule{.1pt}
    \begin{multicols}{2}
      \scriptsize{
        \begin{minted}[autogobble, linenos, breaklines]{python}
          def is_feasible(N, B, T):
          """Testing feasibility of linear program.

          Args:
            N, B (list[int]): lists of non-basic and
            basic variables.
            T (ndarray[float]): numpy array for
            tableau of linear program.
          Output:
            (bool) True if program is feasible
            False otherwise.
          Boundary effect:
            Transforms T into an equivalent LP having
            basic feasible solution if T is feasible.
          """
          p, q = T.shape[0], T.shape[1]
          c = T[0, :]
          T[0, :] = [0]*q
          new_c = np.array([1] + [-1]*(p-1), dtype=float)
          T = np.insert(T, -1, new_c, axis=1)
          N.append(0)
          # pivoting to be basic feasible
          i_min = np.argmin(T[1: ,-1])
          pivot(N, B, T, i_min, 0)
          V = _simplex(N, B, T)
          if 0 in B:
            l, e = B.index(0), 0
            while T[l, e] == 0 or e == l:
              e += 1
            pivot(N, B, T, l, e)
          for i in B:
            c -= c[i]*T[B.index(i), :]
          T[0, :] = c
          return V[0] == 0
      \end{minted}
    }
  \end{multicols}
\end{overlayarea}
\end{frame}

\begin{frame}{Where We Stand?}
  We are now able to
  \begin{itemize}
  \item<1->
    know whether a linear program is feasible or not using
    \mintinline{python}{is_feasible}
  \item<2->
    if it is feasible we can build up an equivalent linear program
    which has feasible basic solution using
    \mintinline{python}{init_simplex}
  \item<3->
    once we get a linear program having feasible basic solution we can
    check whether our linear program is unbounded or has a
    \textit{hopefully} optimal finite solution using the
    \textit{restricted} \mintinline{python}{_simplex}.
  \end{itemize}
  \pause[4]
  We are left with showing validity of \textit{restricted}
  \mintinline{python}{_simplex}!
  \begin{tcolorbox}[
    enhanced,
    parbox = false,
    colback = mLightBrown!30!white,
    colframe = mLightBrown!30!white,
    arc = 0mm,
    ]
    We call \mintinline{python}{simplex} the algorithm globally going
    through all previous steps.
  \end{tcolorbox}
\end{frame}

\begin{frame}
  \centering
  {\huge \textbf{That's it for today !}}
\end{frame}

\end{document}

%%% Local Variables:
%%% mode: latex
%%% TeX-master: t
%%% End:
