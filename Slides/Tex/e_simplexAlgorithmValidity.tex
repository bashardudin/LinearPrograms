\documentclass[32pt, aspectratio = 169]{beamer}

\usepackage[utf8]{inputenc} % Character encoding.

\pdfinfo{
   /Author (Bashar Dudin)
   /Title  (Validity of the Simplex Algorithm)
   /Subject (Linear Programs)
}


\usepackage{./Style/linearProgramsBeamer} % This is extra styling for Beamer environments.
\usepackage{./Style/linearProgramsStyle} % This is a set of commands for maths content.

%-------------------------------------------------------------------------------
%   TITLE PAGE
%-------------------------------------------------------------------------------

\author[BD]{Bashar Dudin}

\institute[]{EPITA}

\title{Linear Programs} %
\subtitle{Validity of the Simplex Algorithm}

%-------------------------------------------------------------------------------
%   DOCUMENT BODY
%-------------------------------------------------------------------------------

\begin{document}

\begin{frame}[plain]
    \titlepage % Print the title page as the first slide
\end{frame}

\begin{frame}{Where We Stand? What Is Left?}
  Given a linear program we are now able to check whether it is
  feasible or not. Under such an assumption we can find an equivalent
  linear program who has basic feasible solution. When that is the
  case we can execute the \textit{restricted} simplex algorithm. We
  find ourselves in front of the upcoming two outputs
    \begin{itemize}
        \item the program is unbounded
        \item the algorithm terminates with a finite output.
    \end{itemize}
    Up so far we haven't proved that this procedure gives back an
    optimal solution. We only know it is feasible. Showing optimality
    is the subject of this set of slides.
\end{frame}

\begin{frame}{The Dual Linear Program}
    \begin{columns}[T]
        \begin{column}{.5\textwidth}
            ~\vspace{.15\baselineskip}

            Let $L$ be the linear program (called \emph{primal}),
            given in standard form by :
            \begin{figure}
            \begin{linearProgG}{
                ${\displaystyle z = \sum_{j = 1 }^n c_jx_j}$
                }{
                ${\displaystyle \forall i \in \{1, \ldots, m\} , \quad \sum_{j = 1}^n a_{ij}x_j \leq b_i}$
                }{
                $\forall j \in \{1, \ldots, n\}, \quad x_j \geq 0$
                }
            \end{linearProgG}
            \end{figure}
        \end{column}
        \begin{column}{.55\textwidth}
            \begin{tcolorbox}[
                enhanced,
                parbox = false,
                colback = mLightBrown!30!white,
                colframe = mLightBrown!30!white,
                arc = 0mm,
                ]
                Out of $L$ we build the following linear program
                $L^\vee$ called the \emph{dual} of $L$ :
                \begin{figure}
                   \begin{linearProg}{
                        minimize
                        }{
                        ${\displaystyle z = \sum_{i=1}^m b_iy_i }$
                        }{
                        ${\displaystyle \forall j \in \{1, \ldots, n\}, \quad \sum_{i =1 }^m a_{ij}y_i \geq c_j}$
                        }{
                        $\forall i \in \{1, \ldots, m\}, \quad y_i \geq 0$
                        }
                    \end{linearProg}
                \end{figure}
            \end{tcolorbox}
        \end{column}
    \end{columns}
\end{frame}

\begin{frame}{Duality}
    The main result justifying, for us, the existence and denomination of the dual linear program is the following.
    \begin{alertblock}{Duality}
        A linear program $L$ and its dual $L^\vee$ have same optimal objective values if bounded.
    \end{alertblock}
    We have, in fact, a much more precise statement :
    \begin{prop}
      If $L$ is in a slack form having optimal basic solution then
      setting $y_i = -c_{n+i}$ gives an optimal solution of $L^\vee$.
    \end{prop}
\end{frame}

\begin{frame}{Duality}
  We'll only be checking duality and giving a glimpse of the
  perception to have.

  Assume we are given a feasible solution
  $(\bar{x}_1, \ldots, \bar{x}_n)$ of $L$ and another one
  $(\bar{y}_1, \ldots, \bar{y}_m)$ of $L^\vee$. Using constraints of
  both linear programs we have that
    \begin{align}
    \sum_{i=1}^m b_i\bar{y}_i & \geq \sum_{i=1}^m \sum_{j=1}^n a_{ij}\bar{x}_j\bar{y}_i \\
        & \geq \sum_{j=1}^n \sum_{i=1}^m a_{ij}\bar{y}_i\bar{x}_j \geq \sum_{j=1}^n c_j\bar{x}_j
\end{align}
    \begin{tcolorbox}[
        enhanced,
        parbox = false,
        colback = mLightBrown!30!white,
        colframe = mLightBrown!30!white,
        arc = 0mm,
        ]
        The optimal objective value of $L$, if any, is bounded from
        above by the objective value of any feasible solution of
        $L^\vee$. Conversely, the optimal objective value of $L^\vee$
        is bounded from below by the objective value of any feasible
        solution of $L$.
    \end{tcolorbox}
\end{frame}

\begin{frame}{Duality}
    The previous reasoning stands for the following result
    \begin{lem}
      If $L$ and $L^\vee$ have respective feasible solutions
      $(\bar{x}_1, \ldots, \bar{x}_n)$ and
      $(\bar{y}_1, \ldots, \bar{y}_n)$ such that
        \[
        \sum_{i=1}^m b_i\bar{y}_i = \sum_{j=1}^nc_j\bar{x}_j
        \]
        then both are \emph{optimal} feasible solutions of their respective linear programs.
    \end{lem}
    \begin{demo}
      We've just seen that the optimal objective value of $L$ is
      smaller or equal to the one of $L^\vee$. The previous hypothesis
      just says it also has to be bigger or equal to the one of
      $L^\vee$, thus equality.
    \end{demo}
\end{frame}

\begin{frame}{Duality}
  To prove the duality theorem as well as the subsequent proposition
  the strategy is the following :
    \begin{itemize}
        \item
          run \textsc{Simplex} on $L$
        \item
          if result is a solution having finite\footnote{If it is
            unfeasible or unbounded above lemma ensures this is also
            the case of $L^\vee$.} objective value, check that
          instruction of proposition gives a feasible solution of
          $L^\vee$ having same objective value
        \item
          lemma implies both objective values are optimal.
    \end{itemize}
    This reasoning shows that if we have an optimal feasible solution
    for $L$ then we have one of same objective value for $L^\vee$. To
    check the converse try making sense of
    \begin{halfshyblock}{Bi-duality}
        The dual of the dual linear program is the primal one.
    \end{halfshyblock}
\end{frame}

\begin{frame}{Fundamental Theorem of Linear Programming}
  Incidently, the previous strategy shows that \textsc{Simplex} finds
  solutions of both $L$ and $L^\vee$ that have same objective
  values. Thus \textsc{Simplex} does compute an optimal solution of
  $L$.
    \begin{thm}
      \textsc{Simplex} does either check whether a linear program $L$
      is feasible, unbounded or gives back an optimal feasible
      solution of $L$ having finite objective value.
    \end{thm}
    This gives the following core result on linear programs :
    \begin{cor}[\textit{(Fundamental Theorem of Linear Programming)}]
      Any linear program $L$ is either infeasible, unbounded or has
      finite optimal objective value
    \end{cor}
\end{frame}


\begin{frame}
        \centering
        {\huge \textbf{Merry Christmas!}}
\end{frame}

\end{document}

%%% Local Variables:
%%% mode: latex
%%% TeX-master: t
%%% End:
